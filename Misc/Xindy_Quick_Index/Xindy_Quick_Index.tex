% !TEX TS-program = XeLaTeX
% Commands for running this example:
% 	 xelatex --shell-escape -interaction=nonstopmode -synctex=-1 Xindy_Quick_Index
%      xindy -L persian -C utf8 -I xindy -M Xindy_Quick_Index.xdy -t Xindy_Quick_Index.glg -o Xindy_Quick_Index.gls Xindy_Make_Glossaries.glo
% 	 xelatex --shell-escape -interaction=nonstopmode -synctex=-1 Xindy_Quick_Index
% End of Commands
\documentclass[a4paper]{article}
\pagestyle{empty}
\renewcommand{\baselinestretch}{0.9} 
\usepackage{geometry}\geometry{left=35mm,right=35mm,top=25mm,bottom=35mm}
\usepackage[colorlinks]{hyperref}
\usepackage{makeidx}
\makeindex
\usepackage[quickindex]{xepersian}

\begin{document}
\عنوان{\lr{Quick Index} \\ ساخت نمایه  با یک بار اجرای \XeLaTeX \\ }
\نویسنده{\url{http://forum.parsilatex.com/}}
\date{}
\عنوان‌ساز
در این نوشتار  تولید نمایه به روشی را توضیح خواهیم داد که بدون زدن دکمه‌‌ای جداگانه برای \lr{Index}سازی و هیچ دستور دیگری و کار اضافه‌ی دیگری، نمایه به صورت اتوماتیک همزمان با اجرای \XeLaTeX روی فایل‌تان ایجاد خواهد شد.
\شروع{شمارش}
\فقره 
دستور \lr{Quick Build}  را به دستور زیر تغییر دهید.
\begin{LTR}\begin{verbatim}
xelatex --shell-escape -interaction=nonstopmode -synctex=-1 %.tex
\end{verbatim}\end{LTR}
نکته: اگر از این فایل \lr{TeX} این دستور بالا را در تک‌میکر کپی کنید، کار نخواهد کرد. چون تک‌میکر یک ویرایشگر دو جهته است و کاراکترهای کنترلی یونیکد اضافه می‌کند. در کنار همین فایل، در فایل \lr{readme.txt} دستور را گذاشته‌ایم. از داخل آن کپی کنید و در \lr{Quick Build} بچسبانید. یا اینکه روی خط بالا کلیک راست کرده و \lr{Remove Unicode Control Character} را بزنید و سپس \lr{copy} کنیدو
\فقره
بسته‌ی \زی‌پرشین را به صورت 
\verb|\usepackage[quickindex]{xepersian}|
 فراخوانی کنید و فایل را با \XeLaTeX اجرا کنید.
 \پایان{شمارش}
\زیرقسمت*{نکات:}
\شروع{شمارش}
\فقره 
راهنمای تغییر دستور \lr{Quick Build}
\شروع{فقرات}
\فقره تک‌میکر را باز کنید. 
\فقره منوی \lr{Options} بروید. 
\فقره گزینه \lr{Configure TeXMaker}  را بزنید. 
\فقره به \lr{Quick Build}  از پانل سمت چپ بروید. 
\فقره دستور گفته شده را در آخرین کادر که \lr{user}  هست را جایگزین قبلی کنید.
\پایان{فقرات}
\فقره
این کار یعنی (تولید نمایه با \lr{Xindy}) بهتر از \lr{Make Index} هست. چون \lr{Make Index} با ترتیب بعضی حروف مشکل داره.
\فقره
در صورت استفاده از این روش حتما آنتی‌ویروس خود را آپدیت نگه دارید. چون  استفاده از \lr{write18} یک زمینه برای ورود ویروس هست.
\پایان{شمارش}



\index{کتاب}
\index{پارسی‌لاتک}
\index{بی‌دی}
\index{سوال}
\index{عنصر}
\index{گزینه}
\index{ژاکت}
\index{مرکز دانلود}
\index{اجرا}
\index{تک‌لایو}
\index{ثالث}
\index{جهان}
\index{چهار}
\index{حمایت}
\index{خواهش}
\index{دنیا}
\index{زی‌پرشین}
\index{ریحان}
\index{شیرین}
\index{صمیمی}
\index{ضمیر}
\index{طبیب}
\index{ظاهر}
\index{غریب}
\index{قوی}
\index{لاتک}
\index{نان}
\index{وحید}
\index{همه}
\index{یک}

\printindex
\end{document}