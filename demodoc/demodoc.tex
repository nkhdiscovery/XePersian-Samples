% !TEX TS-program = XeLaTeX 
% Commands for running this example:
% xelatex demodoc
% End of Commands
\documentclass[a4paper,12pt]{article}
\pagestyle{empty}
\renewcommand{\baselinestretch}{1.5}
\usepackage{mathpazo}
\usepackage{graphicx}
\usepackage{amsmath}
\usepackage[margin=1.5cm,
    vmargin={1.5cm,1.5cm},
    nohead]{geometry}
\usepackage{xepersian}
\title{سند نمونه}
\author{نام نویسنده\\
        در خانه}


\begin{document}

\maketitle

\section{مقدمه}

\subsection{متن نمونه}
اولین قسمت حاوی مقداری متن نمونه است.
در پرونده ورودی زی‌پرشین، خط‌ها می‌توانند هر جایی که شما دوست دارید شکسته شوند،
یا حتی می‌توانید فاصله بیشتری قرار دهید تا متن ورودی زی‌پرشین خواناتر باشد،
اما این تأثیری در خروجی نخواهد گذاشت و اگر یک یا یک‌میلیون فاصله بین کلمات در پرونده ورودی زی‌پرشین قرار دهید،
 تنها یک فاصله در خوجی ظاهر می‌شود.

برای ایجاد یک پاراگراف جدید، کافی است یک خط خالی بگذارید.

\subsection{مشخص کردن نکات جالب متن}

شما می‌توانید برای تأگید متن، آن را بصورت
\emph{ایتالیک} 
یا
 \textbf{سیاه}
حروف‌چینی کنید.

\begin{quote}
همچنین می‌توانید با استفاده از محیط
\texttt{quote}
متن خود را با مقداری حاشیه از هر دو طرف راست و چپ، حروف‌چینی کنید.
\end{quote}

\section{موارد اصلی}

\subsection{ریاضیات}

قوانین خاصی برای حروف‌چینی ریاضی وجود دارد و همچنین تعداد زیادی از دستورات تنها در محیط ریاضی کار می‌کنند.
یک فرمول ریاضی بین متن مانند
$x^2+\beta$ 
دارای فاصله خودکار میان متغیرها و سایر نشانه‌های ریاضی است اما فرمول‌های نمایشی مانند 
\begin{equation}
  \vec{A} \times (\vec{B}\times\vec{C}) =
  (\vec{A}\cdot\vec{C})\vec{B} -
  (\vec{A}\cdot\vec{B})\vec{C}
\end{equation}
\begin{equation}
  \vec{A}\cdot(\vec{B}\times\vec{C}) =
  \begin{vmatrix}
    A_x & A_y & A_y \\
    B_x & B_y & B_y \\
    C_x & C_y & C_y
  \end{vmatrix}
\end{equation}
دارای شماره فرمول خودکار هستند.

\subsection{فهرست‌کردن مطالب}

\begin{enumerate}
\item می‌توانید مطالب را بصورت‌های متفاوتی فهرست کنید، مانند شمارش کردن،
\item به اقلام نوشتن،
\item و یا فهرست توصیفی ایجاد کردن.
\end{enumerate}

\subsection{اضافه‌کردن تصاویر}

تصاویر و نگاره‌سازی‌هایی
\includegraphics[width=1cm]{demo}
که با نرم‌افزارهای دیگر تولید شده‌اند، به راحتی قابل درج شدن می‌باشند.
\end{document}
