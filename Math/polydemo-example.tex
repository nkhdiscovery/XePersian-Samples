% !TEX TS-program = XeLaTeX
% Commands for running this example:
% 	 xelatex polydemo-example
% End of Commands
\documentclass[12pt]{article}
\pagestyle{empty}
\usepackage{geometry,hyperref,color,polynom}
\geometry{screen}
\hypersetup{pdfpagemode=FullScreen}
\usepackage{xepersian}
%%%%%%%%%%%%%
%
% \tokentotext<csname>{<tokens>}
% converts token sequence into a \texttt-printable form.
%
%%%%%%%%%%%%%
\makeatletter % We'll use three internal LaTeX definitions
\newcommand\tokentotext[2]{%
    \def#1{#2}%
    \edef#1{\expandafter\strip@prefix\meaning#1}%
    \edef#1{\expandafter\zap@space#1 \@empty}}
\makeatother


\newcounter{currentstage}
%%%%%%%%%%%%%
%
% \dopolylongdiv{<page title>}{<loop min>}{<loop max>}{<key=value list for \polyset>}
%
%%%%%%%%%%%%%
\newcommand\dopolylongdiv[4]{%
  \begingroup
    \tokentotext\keyvaluetext{#4}%
    \polyset{#4}%
    \setcounter{currentstage}{#2}\addtocounter{currentstage}{-1}%
    %
    \loop \ifnum \value{currentstage}<#3\relax
        \addtocounter{currentstage}{1}%
        \newpage
        \section*{\strut#1}\section*{\strut\texttt{\keyvaluetext}}%
        \[\polylongdiv[stage=\value{currentstage}]{x^3+x^2-1}{x+2}\]
    \repeat
  \endgroup}
%%%%%%%%%%%%%
%
% \dopolyhornerscheme{<page title>}{<loop min>}{<loop max>}{<key=value list for \polyset>}
%
%%%%%%%%%%%%%
\newcommand\dopolyhornerscheme[4]{% similar to \dopolylongdiv
  \begingroup
    \tokentotext\keyvaluetext{#4}%
    \polyset{#4}%
    \setcounter{currentstage}{#2}\addtocounter{currentstage}{-1}%
    %
    \loop \ifnum \value{currentstage}<#3\relax
        \addtocounter{currentstage}{1}%
        \newpage
        \section*{\strut#1}\section*{\strut\texttt{\keyvaluetext}}%
        \[\polyhornerscheme[x=-2,stage=\value{currentstage}]{x^3+x^2-1}\]
    \repeat
  \endgroup}

%%%%%%%%%%%%%
%
% T H E   D O C U M E N T
%
%%%%%%%%%%%%%
\begin{document}

\LARGE

\dopolylongdiv{تقسیم چند‌جمله‌ای‌ها -- مرحله ۱ تا ۱۰}{1}{10}
              {}

\dopolylongdiv{تقسیم چند‌جمله‌ای‌ها -- مرحله ۱ تا ۱۱}{1}{11}
              {style=B}

\dopolylongdiv{تقسیم چندجمله‌ای‌ها -- مرحله ۱ تا ۱۱}{1}{11}
              {style=C}

\dopolyhornerscheme{روش هورنر -- مرحله ۱ تا ۸}{1}{8}
                   {}

\dopolyhornerscheme{روش هورنر -- مرحله ۱ تا ۸}{1}{8}
                   {tutor=true,resultstyle=\color{blue}}

\dopolyhornerscheme{روش هورنر -- قاعده و نتیجه}{8}{8}
                   {resultbottomrule,resultleftrule,resultrightrule}

\dopolyhornerscheme{روش هورنر -- مرحله ۱ تا ۸}{1}{8}
                   {tutor=true,tutorlimit=3}

\dopolyhornerscheme{روش هورنر: انتخاب‌های بیشتر}{8}{8}
                   {showbase=top,showbasesep=false}

\end{document}
