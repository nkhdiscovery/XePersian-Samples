% !TEX TS-program = XeLaTeX 
% Commands for running this example:
% xelatex autobaselineskip-example
% End of Commands
\documentclass{article}
\pagestyle{empty}
\usepackage{xepersian}
\راحت
\شمارجدید\شمارفاصله‌کرسی
\بعدجدید\چسب‌بین‌سطرها
\بعدجدید\چسب‌بین‌سطرهای‌شمع
\کادرجدید\وکادر
\کادرجدید\وکادرشمع
\تر\فاصله‌کرسی‌خودکار{% در اینجا ما فاصله کرسی مناسب را محاسبه می‌کنیم
 \شمارفاصله‌کرسی=0\چسب‌بین‌سطرها=0pt\چسب‌بین‌سطرهای‌شمع=0pt
 \حلقه\درکادر\وکادر=\کادرا{\نویسه\شمارفاصله‌کرسی}
 \گربعد\ارتفاع\وکادر>\چسب‌بین‌سطرها\چسب‌بین‌سطرها=\ارتفاع\وکادر\رگ
 \گربعد\عمق\وکادر>\چسب‌بین‌سطرهای‌شمع\چسب‌بین‌سطرهای‌شمع=\عمق\وکادر\رگ
 \بیفزابر\شمارفاصله‌کرسی  1
 \گرعدد\شمارفاصله‌کرسی<128\ازنو
 \ارتفاع\وکادر=\چسب‌بین‌سطرها
 \عمق\وکادر=\چسب‌بین‌سطرهای‌شمع
 \چسب‌بین‌سطرهای‌شمع=\ارتفاع\وکادر %\بیفزابر\چسب‌بین‌سطرهای‌شمع \چسب‌بین‌سطرها
 \گربعد\چسب‌بین‌سطرهای‌شمع>\فاصله‌کرسی
  \پیام{من نویسه‌ای را در قلمی که استفاده می‌کنید، پیدا کردم که ارتفاعش بیشتر از اندازه فاصله کرسی معمول است.}
  \پیام{بنابراین من مقدار فاصله کرسی را به ۱٫۵ برابر ارتفاع این نویسه افزایش می‌دهم!}
  \فاصله‌کرسی=1.5\چسب‌بین‌سطرهای‌شمع
 \رگ
 \عام\چسب‌بین‌سطرها=\فاصله‌کرسی
 \عام\بیفزابر\چسب‌بین‌سطرها-\ارتفاع\وکادر \عام\بیفزابر\چسب‌بین‌سطرها-\عمق\وکادر
 \چسب‌بین‌سطرهای‌شمع=\ارتفاع\وکادر \بیفزابر\چسب‌بین‌سطرهای‌شمع \چسب‌بین‌سطرها
 \درکادر\وکادرشمع=\کادرا{\خط‌و height\چسب‌بین‌سطرهای‌شمع depth\عمق\وکادر width0pt}}
\شروع{نوشتار}
\قسمت*{فاصله کرسی نامناسب}
شاهنامه شرح احوال، پیروزیها، شکستها، ناکامیها و دلاوریهای ایرانیان از کهن‌ترین دوران (نخستین پادشاه جهان کیومرث) تا سرنگونی دولت ساسانی به دست تازیان است (در سده هفتم میلادی). کشمکشهای خارجی ایرانیان با هندیان در شرق، تورانیان در شرق و شمال شرقی، رومیان در غرب و شمال غربی و تازیان در جنوب غربی است. علاوه بر سیر خطی تاریخی ماجرا، در شاهنامه داستان‌های مستقل پراکنده‌ای نیز وجود دارند که مستقیماً به سیر تاریخی مربوط نمی‌شوند. از آن جمله: داستان زال و رودابه، رستم و سهراب، بیژن و منیژه، بیژن و گرازان(که بخشی از داستان بلند بیژن و منیژه است)، کرم هفتواد و جز اینها بعضی از این داستان‌ها به طور خاص چون رستم و اسفندیار و یا رستم و سهراب از شاهکارهای مسلم ادبیات جهان به شمار می‌آیند.

\پرش‌بلند
\فاصله‌کرسی‌خودکار
\قسمت*{فاصله کرسی مناسب}
شاهنامه شرح احوال، پیروزیها، شکستها، ناکامیها و دلاوریهای ایرانیان از کهن‌ترین دوران (نخستین پادشاه جهان کیومرث) تا سرنگونی دولت ساسانی به دست تازیان است (در سده هفتم میلادی). کشمکشهای خارجی ایرانیان با هندیان در شرق، تورانیان در شرق و شمال شرقی، رومیان در غرب و شمال غربی و تازیان در جنوب غربی است. علاوه بر سیر خطی تاریخی ماجرا، در شاهنامه داستان‌های مستقل پراکنده‌ای نیز وجود دارند که مستقیماً به سیر تاریخی مربوط نمی‌شوند. از آن جمله: داستان زال و رودابه، رستم و سهراب، بیژن و منیژه، بیژن و گرازان(که بخشی از داستان بلند بیژن و منیژه است)، کرم هفتواد و جز اینها بعضی از این داستان‌ها به طور خاص چون رستم و اسفندیار و یا رستم و سهراب از شاهکارهای مسلم ادبیات جهان به شمار می‌آیند.
\پایان{نوشتار}