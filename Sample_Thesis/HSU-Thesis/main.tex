% !TEX TS-program = XeLaTeX
% Commands for running this example:
% 	 xelatex main
% 	 bibtex8 -W -c cp1256fa main
%    xindy -L persian -C utf8 -M texindy main
% 	 xelatex main
% 	 xelatex main
% End of Commands

%        نمونه پایان‌نامه آماده شده با استفاده از کلاس HSU-Thesis، نگارش ۰.۸
% 		محمود امین‌طوسی، دانشگاه حکیم سبزواری، http://profs.hsu.ac.ir/mamintoosi/
% 		گروه پارسی‌لاتک  http://www.parsilatex.com
%        این نسخه، بر اساس نسخه‌ 0.62 از کلاس IUST-Thesis آماده شده است.
%        
%        تغییرات:
%	   نسخه ۰.۸: تغییرات مختصر و استفاده از قلمهای سری HM
%	   نسخه ۰.۷:  اضافه کردن listings برای درج کد
%        نسخه ۰.۶۲: اصلاح intro در main
%        نسخه ۰.۶۱: درج کد لازم برای تبدیل آ به الف در commands
%        نسخه ۰.۶۰: اصلاح مشکل با بسته subfig 
%----------------------------------------------------------------------------------------------
%        اگر قصد نوشتن پروژه کارشناسی را دارید، در خط زیر به جای msc، کلمه bsc و اگر قصد نوشتن پروژه دکترا
%        را دارید، کلمه phd را قرار دهید. کلیه تنظیمات لازم، به طور خودکار، اعمال می‌شود.

%        اگر مایلید پایان‌نامه شما دورو باشد به جای oneside  در خط زیر از twoside استفاده کنید
\documentclass[oneside,openany,msc,12pt]{HSU-Thesis}

% مشخصات پایان‌نامه را در فایلهای faTitle و enTitle وارد نمایید.

%       فایل commands.tex را مطالعه کنید؛ چون دستورات مربوط به فراخوانی بسته زی‌پرشین 
%       و دیگر بسته‌ها و ... در این فایل قرار دارد و بهتر است که با نحوه استفاده از آنها آشنا شوید.
% !TEX TS-program = XeLaTeX
% !TeX root=main.tex
% در این فایل، دستورها و تنظیمات مورد نیاز، آورده شده است.
%-------------------------------------------------------------------------------------------------------------------

% در ورژن جدید زی‌پرشین برای تایپ متن‌های ریاضی، این سه بسته، حتماً باید فراخوانی شود
\usepackage{amsthm,amssymb,amsmath}
% بسته‌ای برای تنطیم حاشیه‌های بالا، پایین، چپ و راست صفحه
\usepackage[top=30mm, bottom=30mm, left=25mm, right=35mm]{geometry}
% بسته‌‌ای برای ظاهر شدن شکل‌ها و تصاویر متن
\usepackage{graphicx}
% بسته‌ای برای رسم کادر
%\usepackage{framed} 
% بسته‌‌ای برای چاپ شدن خودکار تعداد صفحات در صفحه «معرفی پایان‌نامه»
%\usepackage{lastpage}
% بسته‌ و دستوراتی برای ایجاد لینک‌های رنگی با امکان جهش
\usepackage[pagebackref=false,colorlinks,linkcolor=blue,citecolor=blue]{hyperref}
% چنانچه قصد پرینت گرفتن نوشته خود را دارید، خط بالا را غیرفعال و  از دستور زیر استفاده کنید چون در صورت استفاده از دستور زیر‌‌، 
% لینک‌ها به رنگ سیاه ظاهر خواهند شد که برای پرینت گرفتن، مناسب‌تر است
%\usepackage[pagebackref=false]{hyperref}
% بسته‌ لازم برای تنظیم سربرگ‌ها
\usepackage{fancyhdr}
%
\usepackage{setspace}
\usepackage{algorithm}
\usepackage{algorithmic}
\usepackage{subfigure}
\usepackage[subfigure]{tocloft}

% بسته‌ای برای ظاهر شدن «مراجع» و «نمایه» در فهرست مطالب
\usepackage[nottoc]{tocbibind}
% دستورات مربوط به ایجاد نمایه
\usepackage{makeidx}
\makeindex
% رنگهای موردنیاز در کدنویسی
\usepackage[usenames,dvipsnames]{xcolor}
% بسته مورد نیاز برای نوشتن کدهای برنامه نویسی در نوشتار 
\usepackage{listings}
% بسته موردنیاز برای رسم نمودارهای زیبا
\usepackage{tikz}

%%%%%%%%%%%%%%%%%%%%%%%%%%
% فراخوانی بسته زی‌پرشین و تعریف قلم فارسی و انگلیسی
\usepackage{xepersian}
\settextfont[Scale=1.2]{HM XNiloofar}
\setlatintextfont[Scale=0.9]{Times New Roman}

%%%%%%%%%%%%%%%%%%%%%%%%%%
% چنانچه مایلید اعداد فرمولها با قلمی به جز Persian Modern حروف‌چینی شوند، خط زیر را فعال و قلم موردنظر خود را مشخص کنید.
%\setdigitfont[Scale=1]{HM XNiloofar}
%%%%%%%%%%%%%%%%%%%%%%%%%%
% تعریف قلم‌های فارسی و انگلیسی اضافی برای استفاده در بعضی از قسمت‌های متن
\defpersianfont\titlefont[Scale=1]{HM XTitr}
\setiranicfont[Scale=1]{HM XNiloofar Oblique}				% ایرانیک، خوابیده به راست
% \defpersianfont\nastaliq[Scale=1.2]{IranNastaliq}

%%%%%%%%%%%%%%%%%%%%%%%%%%
% دستوری برای حذف کلمه «چکیده»
\renewcommand{\abstractname}{}
% دستوری برای حذف کلمه «abstract»
%\renewcommand{\latinabstract}{}
% دستوری برای تغییر نام کلمه «اثبات» به «برهان»
\renewcommand\proofname{\textbf{برهان}}
% دستوری برای تغییر نام کلمه «کتاب‌نامه» به «مراجع»
\renewcommand{\bibname}{مراجع}
% دستوری برای تعریف واژه‌نامه انگلیسی به فارسی
\newcommand\persiangloss[2]{#1\dotfill\lr{#2}\\}
% دستوری برای تعریف واژه‌نامه فارسی به انگلیسی 
\newcommand\englishgloss[2]{#2\dotfill\lr{#1}\\}
% تعریف دستور جدید «\پ» برای خلاصه‌نویسی جهت نوشتن عبارت «پروژه/پایان‌نامه/رساله»
\newcommand{\پ}{پروژه/پایان‌نامه/رساله }

%\newcommand\BackSlash{\char`\\}

%%%%%%%%%%%%%%%%%%%%%%%%%%
\SepMark{-}

% تعریف و نحوه ظاهر شدن عنوان قضیه‌ها، تعریف‌ها، مثال‌ها و ...
\theoremstyle{definition}
\newtheorem{definition}{تعریف}[section]
\theoremstyle{theorem}
\newtheorem{theorem}[definition]{قضیه}
\newtheorem{lemma}[definition]{لم}
\newtheorem{proposition}[definition]{گزاره}
\newtheorem{corollary}[definition]{نتیجه}
\newtheorem{remark}[definition]{ملاحظه}
\theoremstyle{definition}
\newtheorem{example}[definition]{مثال}

%\renewcommand{\theequation}{\thechapter-\arabic{equation}}
%\def\bibname{مراجع}
\numberwithin{algorithm}{chapter}
\def\listalgorithmname{فهرست الگوریتم‌ها}
\def\listfigurename{فهرست تصاویر}
\def\listtablename{فهرست جداول}

%%%%%%%%%%%%%%%%%%%%%%%%%%%%
% دستورهایی برای سفارشی کردن سربرگ صفحات
% \newcommand{\SetHeader}{
% \csname@twosidetrue\endcsname
% \pagestyle{fancy}
% \fancyhf{} 
% \fancyhead[OL,EL]{\thepage}
% \fancyhead[OR]{\small\rightmark}
% \fancyhead[ER]{\small\leftmark}
% \renewcommand{\chaptermark}[1]{%
% \markboth{\thechapter-\ #1}{}}
% }
%%%%%%%%%%%%5
%\def\MATtextbaseline{1.5}
%\renewcommand{\baselinestretch}{\MATtextbaseline}
\doublespacing
%%%%%%%%%%%%%%%%%%%%%%%%%%%%%
% دستوراتی برای اضافه کردن کلمه «فصل» در فهرست مطالب

\newlength\mylenprt
\newlength\mylenchp
\newlength\mylenapp

\renewcommand\cftpartpresnum{\partname~}
\renewcommand\cftchappresnum{\chaptername~}
\renewcommand\cftchapaftersnum{:}

\settowidth\mylenprt{\cftpartfont\cftpartpresnum\cftpartaftersnum}
\settowidth\mylenchp{\cftchapfont\cftchappresnum\cftchapaftersnum}
\settowidth\mylenapp{\cftchapfont\appendixname~\cftchapaftersnum}
\addtolength\mylenprt{\cftpartnumwidth}
\addtolength\mylenchp{\cftchapnumwidth}
\addtolength\mylenapp{\cftchapnumwidth}

\setlength\cftpartnumwidth{\mylenprt}
\setlength\cftchapnumwidth{\mylenchp}	

\makeatletter
{\def\thebibliography#1{\chapter*{\refname\@mkboth
   {\uppercase{\refname}}{\uppercase{\refname}}}\list
   {[\arabic{enumi}]}{\settowidth\labelwidth{[#1]}
   \rightmargin\labelwidth
   \advance\rightmargin\labelsep
   \advance\rightmargin\bibindent
   \itemindent -\bibindent
   \listparindent \itemindent
   \parsep \z@
   \usecounter{enumi}}
   \def\newblock{}
   \sloppy
   \sfcode`\.=1000\relax}}
   
% اگر مایلید در شماره‌گذاری حرفی و ابجد به جای آ از الف استفاده شود دستورات زیر را فعال کنید.   
% \def\abj@num@i#1{%
  % \ifcase#1\or الف\or ب\or ج\or د%
           % \or ه‍\or و\or ز\or ح\or ط\fi
  % \ifnum#1=\z@\abjad@zero\fi}   
  
  % \def\@harfi#1{\ifcase#1\or الف\or ب\or پ\or ت\or ث\or
% ج\or چ\or ح\or خ\or د\or ذ\or ر\or ز\or ژ\or س\or ش\or ص\or ض\or ط\or ظ\or ع\or غ\or
% ف\or ق\or ک\or گ\or ل\or م\or ن\or و\or ه\or ی\else\@ctrerr\fi}

\makeatother

%%%%%%%%%%%%%%% امکان درج کد در سند
%  در این قسمت تمام ابزارهای مورد نیاز در نوشتن برنامه ها اورده شده 
%  است. با استفاده از این ابزارهای می‌توان برنامه های مورد نیاز را در مستند جای داد.
\lstset{% general command to set parameter(s) 
basicstyle=\small, % print whole listing small
keywordstyle=\color{blue}\bfseries,
% underlined bold black keywords
identifierstyle=, % nothing happens
stringstyle=\ttfamily\color{red},
commentstyle=\color{LimeGreen}, % white comments
stringstyle=\ttfamily\color{red}, % typewriter type for strings
showstringspaces=false} % no special string spaces



\begin{document}
\pagestyle{empty}
\pagenumbering{harfi}
% !TEX TS-program = XeLaTeX
% !TeX root=main.tex
% در این فایل، عنوان پایان‌نامه، مشخصات خود، متن تقدیمی‌، ستایش، سپاس‌گزاری و چکیده پایان‌نامه را به فارسی، وارد کنید.
% توجه داشته باشید که جدول حاوی مشخصات پروژه/پایان‌نامه/رساله و همچنین، مشخصات داخل آن، به طور خودکار، درج می‌شود.
%%%%%%%%%%%%%%%%%%%%%%%%%%%%%%%%%%%%
% دانشگاه خود را وارد کنید
\university{حکیم سبزواری}
% دانشکده، آموزشکده و یا پژوهشکده  خود را وارد کنید
\faculty{دانشکده ریاضی و علوم کامپیوتر}
% گروه آموزشی خود را وارد کنید - فعلاً فعال نیست
%\department{ریاضی کاربردی}
% رشته خود را وارد کنید
\subject{ریاضی کاربردی}
% گرایش خود را وارد کنید
\field{بهینه‌سازی}
% عنوان پایان‌نامه را وارد کنید
\title{نوشتن پروژه، پایان‌نامه و رساله با استفاده از کلاس 
HSU-Thesis}
% نام استاد(ان) راهنما را وارد کنید
\firstsupervisor{وفا خلیقی}
%\secondsupervisor{استاد راهنمای دوم}
% نام استاد(دان) مشاور را وارد کنید. چنانچه استاد مشاور ندارید، دستورات پایین را غیرفعال کنید.
\firstadvisor{محمود امین‌طوسی}
\secondadvisor{وحید دامن‌افشان}
% نام دانشجو را وارد کنید
\name{هادی}
% نام خانوادگی دانشجو را وارد کنید
\surname{صفی‌اقدم}
% شماره دانشجویی دانشجو را وارد کنید
\studentID{89922012}
% تاریخ پایان‌نامه را وارد کنید
\thesisdate{اسفند ۱۳۹۲}
% به صورت پیش‌فرض برای پایان‌نامه‌های کارشناسی تا دکترا به ترتیب از عبارات «پروژه»، «پایان‌نامه» و »رساله» استفاده می‌شود؛ اگر  نمی‌پسندید هر عنوانی را که مایلید در دستور زیر قرار داده و آنرا از حالت توضیح خارج کنید.
%\projectLabel{پایان‌نامه}

% به صورت پیش‌فرض برای عناوین مقاطع تحصیلی کارشناسی تا دکترا به ترتیب از عبارات «کارشناسی»، «کارشناسی ارشد» و »دکترا» استفاده می‌شود؛ اگر  نمی‌پسندید هر عنوانی را که مایلید در دستور زیر قرار داده و آنرا از حالت توضیح خارج کنید.
%\degree{}

\firstPage
\besmPage
\davaranPage

%\vspace{.5cm}
% در این قسمت اسامی اساتید راهنما، مشاور و داور باید به صورت دستی وارد شوند
%\renewcommand{\arraystretch}{1.2}
\begin{center}
\begin{tabular}{| p{8mm} | p{18mm} | p{.17\textwidth} |p{14mm}|p{.2\textwidth}|c|}
\hline
ردیف	& سمت & نام و نام خانوادگی & مرتبه \newline دانشگاهی &	دانشگاه یا مؤسسه &	امضـــــــــــــا\\
\hline
۱  &	استاد راهنما				 & دکتر وفا خلیقی & استادیار & دانشگاه استرالیا &  \\
\hline
۲ &     استاد مشاور				 & دکتر محمود \newline امین‌طوسی& استادیار & دانشگاه \newline حکیم سبزواری& \\
\hline
۳ &     استاد مشاور				 & وحید دامن‌افشان& مربی & دانشگاه \newline صنعتی کرمانشاه& \\
\hline
۴ &      استاد مدعو\newline  خارجی			 & دکتر فرشاد ترابی & استادیار & دانشگاه خواجه نصیر& \\
\hline
۴ &	استاد مدعو\newline  داخلی			 & دکتر مهدی امیدعلی   & استادیار & دانشگاه شاهد& \\
\hline
۵ &	استاد مدعو\newline  داخلی			 & دکتر مصطفی واحدی & استادیار & دانشگاه \newline صنعتی شریف& \\
\hline
\end{tabular}
\end{center}

\esalatPage
\mojavezPage


% چنانچه مایل به چاپ صفحات «تقدیم»، «نیایش» و «سپاس‌گزاری» در خروجی نیستید، خط‌های زیر را با گذاشتن ٪  در ابتدای آنها غیرفعال کنید.
 % پایان‌نامه خود را تقدیم کنید!

 \newpage
\thispagestyle{empty}
{\Large تقدیم به:}\\
\begin{flushleft}
{\huge
همسر و فرزندانم\\
\vspace{7mm}
و\\
\vspace{7mm}
پدر و مادرم
}
\end{flushleft}


% سپاس‌گزاری
\begin{acknowledgementpage}
سپاس خداوندگار حکیم را که با لطف بی‌کران خود، آدمی را زیور عقل آراست.


در آغاز وظیفه‌  خود  می‌دانم از زحمات بی‌دریغ استاد  راهنمای خود،  جناب آقای دکتر ...، صمیمانه تشکر و  قدردانی کنم  که قطعاً بدون راهنمایی‌های ارزنده‌  ایشان، این مجموعه  به انجام  نمی‌رسید.

از جناب  آقای  دکتر ...   که زحمت  مطالعه و مشاوره‌  این رساله را تقبل  فرمودند و در آماده سازی  این رساله، به نحو احسن اینجانب را مورد راهنمایی قرار دادند، کمال امتنان را دارم.

همچنین لازم می‌دانم از پدید آورندگان بسته زی‌پرشین، مخصوصاً جناب آقای  وفا خلیقی، که این پایان‌نامه با استفاده از این بسته، آماده شده است و همه دوستانمان در گروه پارسی‌لاتک کمال قدردانی را داشته باشم.

 در پایان، بوسه می‌زنم بر دستان خداوندگاران مهر و مهربانی، پدر و مادر عزیزم و بعد از خدا، ستایش می‌کنم وجود مقدس‌شان را و تشکر می‌کنم از خانواده عزیزم به پاس عاطفه سرشار و گرمای امیدبخش وجودشان، که بهترین پشتیبان من بودند.
% با استفاده از دستور زیر، امضای شما، به طور خودکار، درج می‌شود.
\signature 
\end{acknowledgementpage}
%%%%%%%%%%%%%%%%%%%%%%%%%%%%%%%%%%%%
% کلمات کلیدی پایان‌نامه را وارد کنید
\keywords{زی‌پرشین، لاتک، قالب پایان‌نامه، الگو}
%چکیده پایان‌نامه را وارد کنید، برای ایجاد پاراگراف جدید از \\ استفاده کنید. اگر خط خالی دشته باشید، خطا خواهید گرفت.
\fa-abstract{
این پایان‌نامه، به بحث در مورد نوشتن پروژه، پایان‌نامه و رساله با استفاده از کلاس 
\lr{HSU-Thesis}
می‌پردازد.
حروف‌چینی پروژه کارشناسی، پایان‌نامه یا رساله یکی از موارد پرکاربرد استفاده از زی‌پرشین است. 
زی‌پرشین بسته‌ای است که به همت آقای وفا خلیقی آماده شده است و امکان حروف‌چینی فارسی در \lr{\LaTeXe}{} را  برای فارسی‌زبانان فراهم کرده است.
از جمله مزایای لاتک آن است که در صورت وجود یک کلاس آماده برای حروف‌چینی یک سند خاص مانند یک پایان‌نامه، کاربر بدون درگیری با جزییات حروف‌چینی و صفحه‌آرایی می‌تواند سند خود را آماده نماید.
\\
شاید با قالب‌های لاتکی که برخی از مجلات برای مقالات خود عرضه می‌کنند مواجه شده باشید. اگر نظیر این کار در دانشگاههای مختلف برای اسناد متنوع آنها مانند پایا‌ن‌نامه‌ها آماده شود، دانشجویان به جای وقت گذاشتن روی صفحه‌آرایی مطالب خود، روی محتوای متن خود تمرکز خواهند نمود. به علاوه با آشنایی با لاتک خواهند توانست از امکانات بسیار این نرم‌افزار جهت نمایش بهتر دست‌آوردهای خود استفاده کنند.
به همین خاطر، یک کلاس با نام 
\lr{HSU-Thesis}
 برای حروف‌چینی پروژه‌ها، پایان‌نامه‌ها و رساله‌های دانشگاه حکیم سبزواری با استفاده از نرم‌افزار زی‌پرشین،  آماده شده است. البته هنوز این فایل به تایید مدیریت تحصیلات تکمیلی دانشگاه حکیم سبزواری نرسیده است. با اینحال دانشجویان می‌توانند از آن استفاده کنند و هنگامی که نسخه‌ی نهایی آماده شد، فقط کافیست تغییرات مختصری روی فایلهای خود بدهند تا مورد تایید دانشگاه قرار گیرد.
}


\abstractPage

\newpage\clearpage
%\setstretch{3} برای تغییر فاصله بین خطوط از این نقطه به بعد
\thispagestyle{empty}
\tableofcontents

\newpage
\listoffigures \newpage
\listoftables  \newpage
\addcontentsline{toc}{chapter}{\listalgorithmname}
\listofalgorithms \newpage
% !TEX TS-program = XeLaTeX
% !TeX root=main.tex
\chapter*{فهرست علائم اختصاری}
\addcontentsline{toc}{chapter}{فهرست علائم اختصاری}

\persiangloss{شتاب گرانش}{$a$ (m/s$^2$)}
\persiangloss{نیرو}{$F$ (N)}


\pagestyle{fancy}
\pagestyle{plain}
\clearpage
\pagenumbering{arabic}

\include{intro}			% فصل اول: مقدمه
\include{latexIntro}		% فصل دوم: آشنایی مقدماتی با لاتک

\pagestyle{fancy}

% مراجع
\pagestyle{empty}
{
\onehalfspacing
\bibliographystyle{acm-fa}%vancouver}
\bibliography{MyReferences}
}

\appendix                           %فصلهای پس از این قسمت به عنوان ضمیمه خواهند آمد.
% اگر شما پیوست اول  خود را در فایلی به جز appendix1 همراه با این کلاس نوشته‌اید باید چندخط اول appendix1 را در فایل خود کپی کنید.
\chapter{‌توپولوژی‌های روی فضاهای اندازه‌ها}
\thispagestyle{empty}
\section{ توپولوژیِ مبهمِ روی فضای اندازه‌ها}
چندین توپولوژی وجود دارد که می‌توان آنها را برای مجموعه اندازه‌ها انتخاب کرد. یک شرط قابل قبول و حداقلی این است که  اگر  تور
 $ (m_{i})_{i\in I} $
به 
$ m $
همگرا باشد آنگاه  باید در 
$\mathbb{R}  $
داشته باشیم 
$\int dm_{i}\to \int fdm  $.
برای مطالعه بیشتر، می‌توان به \cite{mainarticle} مراجعه کرد. از طرف دیگر ...
		% پیوست اول: مدیریت مراجع در لاتک
% !TEX TS-program = XeLaTeX
% !TeX root=main.tex

\chapter{‌جدول، نمودار و الگوریتم در لاتک}\label{App:Latex:More}
\thispagestyle{empty}

در این بخش نمونه مثالهایی از جدول، نمودار و الگوریتم در لاتک را خواهیم دید.
\section{مدلهای حرکت دوبعدی}
بسیاری از اوقات حرکت بین دو تصویر از یک صحنه با یکی از مدلهای پارامتری ذکر شده در جدول \eqref{tab:MotionModels} قابل مدل نمودن می‌باشد.  
\begin{table}[ht]
\caption{مدلهای تبدیل.}
\label{tab:MotionModels}
\centering
\onehalfspacing
\begin{tabular}{|r|c|l|r|}
\hline نام مدل & درجه آزادی & تبدیل مختصات & توضیح \\ 
\hline انتقالی & ۲ & $\begin{aligned} x'=x+t_x \\ y'=y+t_y \end{aligned}$  &  انتقال دوبعدی\\ 
\hline اقلیدسی & ۳ & $\begin{aligned} x'=xcos\theta - ysin\theta+t_x \\ y'=xsin\theta+ycos\theta+t_y \end{aligned}$  &  انتقالی+دوران \\ 
\hline 
\end{tabular} 
\end{table}

\section{ماتریس}

شناخته‌شده‌ترین روش تخمین ماتریس هوموگرافی الگوریتم تبدیل خطی مستقیم (\lr{DLT\LTRfootnote{Direct Linear Transform}}) است.  فرض کنید چهار زوج نقطهٔ متناظر در دو تصویر در دست هستند،  $\mathbf{x}_i\leftrightarrow\mathbf{x}'_i$   و تبدیل با رابطهٔ
  $\mathbf{x}'_i = H\mathbf{x}_i$
  نشان داده می‌شود که در آن:
\[\mathbf{x}'_i=(x'_i,y'_i,w'_i)^\top  \]
و
\[ H=\left[
\begin{array}{ccc}
h_1 & h_2 & h_3 \\ 
h_4 & h_5 & h_6 \\ 
h_7 & h_8 & h_9
\end{array} 
\right]\]
رابطه زیر را برای الگوریتم  \eqref{alg:DLT} لازم دارم.
\begin{equation}\label{eq:DLT_Ah}
\left[
\begin{array}{ccc}
0^\top & -w'_i\mathbf{x}_i^\top & y'_i\mathbf{x}_i^\top \\ 
w'_i\mathbf{x}_i & 0^\top & -x'_i\mathbf{x}_i^\top \\ 
- y'_i\mathbf{x}_i^\top & x'_i\mathbf{x}_i^\top & 0^\top
\end{array} 
\right]
\left(
\begin{array}{c}
\mathbf{h}^1 \\ 
\mathbf{h}^2 \\ 
\mathbf{h}^3
\end{array} 
\right)=0
\end{equation}

\section{درج الگوریتم}
\subsection{الگوریتم با دستورات فارسی}
با مفروضات فوق، الگوریتم \lr{DLT} به صورت نشان داده شده در الگوریتم \eqref{alg:DLT}  خواهد بود.
\begin{algorithm}[t]
\onehalfspacing
\caption{الگوریتم \lr{DLT} برای تخمین ماتریس هوموگرافی.} \label{alg:DLT}
\begin{algorithmic}[1]
\REQUIRE $n\geq4$ زوج نقطهٔ متناظر در دو تصویر 
${\mathbf{x}_i\leftrightarrow\mathbf{x}'_i}$،\\
\ENSURE ماتریس هوموگرافی $H$ به نحوی‌که: 
$\mathbf{x}'_i = H \mathbf{x}_i$.
  \STATE برای هر زوج نقطهٔ متناظر
$\mathbf{x}_i\leftrightarrow\mathbf{x}'_i$ 
ماتریس $\mathbf{A}_i$ را با استفاده از رابطهٔ \ref{eq:DLT_Ah} محاسبه کنید.
  \STATE ماتریس‌های ۹ ستونی  $\mathbf{A}_i$ را در قالب یک ماتریس $\mathbf{A}$ ۹ ستونی ترکیب کنید. 
  \STATE تجزیهٔ مقادیر منفرد \lr{(SVD)}  ماتریس $\mathbf{A}$ را بدست آورید. بردار واحد متناظر با کمترین مقدار منفرد جواب $\mathbf{h}$ خواهد بود.
  \STATE  ماتریس هوموگرافی $H$ با تغییر شکل $\mathbf{h}$ حاصل خواهد شد.
\end{algorithmic}
\end{algorithm}

\subsection{الگوریتم با دستورات لاتین}
الگوریتم \ref{alg:RANSAC} یک الگوریتم با دستورات لاتین است.

\begin{algorithm}[t]
\onehalfspacing
\caption{الگوریتم \lr{RANSAC} برای تخمین ماتریس هوموگرافی.} \label{alg:RANSAC}
\begin{latin}
\begin{algorithmic}[1]
\REQUIRE $n\geq4$ putative correspondences, number of estimations, $N$, distance threshold $T_{dist}$.\\
\ENSURE Set of inliers and Homography matrix $H$.
\FOR{$k = 1$ to $N$}
  \STATE Randomly choose 4 correspondence,
  \STATE Check whether these points are colinear, if so, redo the above step
  \STATE Compute the homography $H_{curr}$ by DLT algorithm from the 4 points pairs,
  \STATE $\ldots$ % الگوریتم کامل نیست
  \ENDFOR
  \STATE Refinement: re-estimate H from all the inliers using the DLT algorithm.
\end{algorithmic}
\end{latin}
\end{algorithm}

\section{درج کد}
درج کد به زبانهای مختلف نیز به سادگی امکان‌پذیر است. شکل 
\ref{fig:Code}
یک قطعه کد \lr{MATLAB} را نشان می‌دهد.
\singlespacing
\begin{figure}
\begin{latin}
\begin{lstlisting}[language=MATLAB,breaklines=true,numbers=left, basicstyle=\ttfamily, numberstyle=\footnotesize, numbersep=10pt, captionpos=b, frame=single, breakatwhitespace=false]
% define a continuous function
f = '4*sin(2*pi*t)';
% plot a figure
ezplot(f);
\end{lstlisting}
\end{latin}
\caption{نمونه کد \lr{MATLAB}}
\label{fig:Code}
\end{figure}
\doublespacing

\section{تصویر}
نمونه تصاویری در بخش قبل دیدیم. دو تصویر شیر کنار هم را هم در شکل \ref{fig:twolion} مشاهده می‌کنید.
\begin{figure}[t]
\centering 
\subfigure[شیر ۱]{ \label{fig:twolion:one}
\includegraphics[width=.3\textwidth]{lion}}
%\hspace{2mm}
\subfigure[شیر ۲]{ \label{fig:twolion:two}
\includegraphics[width=.3\textwidth]{lion}}
\caption{دو شیر}
\label{fig:twolion} %% label for entire figure
\end{figure}

\section{نمودار}
لاتک بسته‌هایی با قابلیت‌های زیاد برای رسم انواع مختلف نمودارها دارد. مانند بسته‌های \lr{Tikz} و  \lr{PSTricks}. توضیح اینها فراتر از این پیوست کوچک است.
% مثالهایی از رسم نمودار را در مجموعه پارسی‌لاتک خواهید یافت. 
\footnote{
نمونه مثالهایی از بسته \lr{Tikz} را می‌توانید در \url{http://www.texample.net/tikz/examples/} ببینید. توصیه می‌کنم که حتماً مثالهایی از برخی از آنها را ببینید.}
یک نمونه نمودار رسم شده با بسته‌ی 
\lr{TikZ}
 در شکل 
\ref{fig:parabola}
نشان داده شده است.
\begin{figure}[t]
\centering
\begin{tikzpicture}[scale=2]
  \shade[top color=blue,bottom color=gray!50] 
      (0,0) parabola (1.5,2.25) |- (0,0);
  \draw (1.05cm,2pt) node[above] 
      {$\displaystyle\int_0^{3/2} \!\!x^2\mathrm{d}x$};

  \draw[style=help lines] (0,0) grid (3.9,3.9)
       [step=0.25cm]      (1,2) grid +(1,1);

  \draw[->] (-0.2,0) -- (4,0) node[right] {$x$};
  \draw[->] (0,-0.2) -- (0,4) node[above] {$f(x)$};

  \foreach \x/\xtext in {1/1, 1.5/1\frac{1}{2}, 2/2, 3/3}
    \draw[shift={(\x,0)}] (0pt,2pt) -- (0pt,-2pt) node[below] {$\xtext$};

  \foreach \y/\ytext in {1/1, 2/2, 2.25/2\frac{1}{4}, 3/3}
    \draw[shift={(0,\y)}] (2pt,0pt) -- (-2pt,0pt) node[left] {$\ytext$};

  \draw (-.5,.25) parabola bend (0,0) (2,4) node[below right] {$x^2$};
\end{tikzpicture}
\caption{یک نمودار زیبا با ارقام فارسی و قابلیت بزرگ‌نمایی بسیار، بدون از دست دادن کیفیت.}
\label{fig:parabola}
\end{figure}

%\baselineskip=.75cm
\onehalfspacing
\addcontentsline{toc}{chapter}{واژه‌نامه فارسی به انگلیسی}
\thispagestyle{empty}
\chapter*{واژه‌نامه فارسی به انگلیسی}
\markboth{واژه‌نامه فارسی به انگلیسی}{واژه‌نامه فارسی به انگلیسی}

\noindent
\englishgloss{probability}{احتمال}
\englishgloss{posterior probability}{احتمال پسینی}
\englishgloss{prior probability}{احتمال پیشینی}
\englishgloss{occurrence probability}{احتمال رخداد}
\englishgloss{propositional probability}{احتمال گزاره‌ای}
\englishgloss{observation probability}{احتمال {\observation}}
\englishgloss{axiom}{اصل موضوع}
\englishgloss{reduction axioms}{اصول موضوعه‌ی {\reduction}}
\englishgloss{public announcement}{اعلان عمومی}
\englishgloss{probability measure}{اندازه‌ی احتمالاتی}
\englishgloss{inner probability measure}{اندازه‌ی احتمالاتی درونی}
\englishgloss{static}{ایستا}
\englishgloss{belief}{باور}
\englishgloss{update}{به‌روزرسانی}
\addcontentsline{toc}{chapter}{واژه‌نامه  انگلیسی به  فارسی}
\thispagestyle{empty}
\chapter*{واژه‌نامه  انگلیسی به  فارسی}
\markboth{واژه‌نامه  انگلیسی به  فارسی}{واژه‌نامه  انگلیسی به  فارسی}

\noindent
\persiangloss{عامل}{agent}
\persiangloss{اصل موضوع}{axiom}
\persiangloss{باور}{belief}
\persiangloss{همه‌دانی مشترک}{common knowledge}
\persiangloss{تمامیت}{completeness}
\persiangloss{سازگار}{consistent}
\persiangloss{جمع‌پذیر شمارا}{countably additive}
\persiangloss{تمییز دادن}{distinguish}
\persiangloss{پویا}{dynamic}
\persiangloss{شناخت}{epistemic}
\persiangloss{عمل}{event}
\persiangloss{یای انحصاری}{exclusive or}
\persiangloss{جمع‌پذیر متناهی}{finitely additive}
\persiangloss{فرمول}{formula}
\persiangloss{فرادانش}{higher order information}
\persiangloss{خودبیمار انگار}{hypochondriac}
\persiangloss{اندازه‌ی احتمالاتی درونی}{inner probability measure}
\persiangloss{خودآگاهی}{intropection}




\printindex
% !TEX TS-program = XeLaTeX
% !TeX root=main.tex
% در این فایل، عنوان پایان‌نامه، مشخصات خود و چکیده پایان‌نامه را به انگلیسی، وارد کنید.

%%%%%%%%%%%%%%%%%%%%%%%%%%%%%%%%%%%%
\baselineskip=.6cm
\begin{latin}
\latinuniversity{Hakim Sabzevari University}
\latinfaculty{Faculty of Mathematics and Computer Science}
\latinsubject{Applied Mathematics}
\latinfield{Optimization}
\latintitle{Writing projects, theses and dissertations using HSU-Thesis Class}
\firstlatinsupervisor{First Supervisor}
%\secondlatinsupervisor{Second Supervisor}
\firstlatinadvisor{First Advisor}
%\secondlatinadvisor{Second Advisor}
\latinname{Vahid}
\latinsurname{Damanafshan}
\latinthesisdate{February 2013}
\latinkeywords{Writing Thesis, Template, \LaTeX, \XePersian}
\en-abstract{
This thesis studies on writing projects, theses and dissertations using HSU-Thesis Class. It ...
}
\latinfirstPage
\end{latin}

\label{LastPage}

\end{document}