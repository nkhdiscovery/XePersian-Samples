% !TEX TS-program = XeLaTeX
% Commands for running this example:
% 	 xelatex main
% 	 bibtex8 -W -c cp1256fa main
% 	 xindy -L persian -C utf8 -M texindy main
% 	 xelatex main
% 	 xelatex main
% End of Commands
\documentclass[a4paper,11pt,msc]{SBU}
% !TEX TS-program = XeLaTeX
% !TeX root=main.tex
% در این فایل، دستورها و تنظیمات مورد نیاز، آورده شده است.
%-------------------------------------------------------------------------------------------------------------------

% در ورژن جدید زی‌پرشین برای تایپ متن‌های ریاضی، این سه بسته، حتماً باید فراخوانی شود
\usepackage{amsthm,amssymb,amsmath}
% بسته‌ای برای تنطیم حاشیه‌های بالا، پایین، چپ و راست صفحه
\usepackage[top=30mm, bottom=30mm, left=25mm, right=35mm]{geometry}
% بسته‌‌ای برای ظاهر شدن شکل‌ها و تصاویر متن
\usepackage{graphicx}
% بسته‌ای برای رسم کادر
%\usepackage{framed} 
% بسته‌‌ای برای چاپ شدن خودکار تعداد صفحات در صفحه «معرفی پایان‌نامه»
%\usepackage{lastpage}
% بسته‌ و دستوراتی برای ایجاد لینک‌های رنگی با امکان جهش
\usepackage[pagebackref=false,colorlinks,linkcolor=blue,citecolor=blue]{hyperref}
% چنانچه قصد پرینت گرفتن نوشته خود را دارید، خط بالا را غیرفعال و  از دستور زیر استفاده کنید چون در صورت استفاده از دستور زیر‌‌، 
% لینک‌ها به رنگ سیاه ظاهر خواهند شد که برای پرینت گرفتن، مناسب‌تر است
%\usepackage[pagebackref=false]{hyperref}
% بسته‌ لازم برای تنظیم سربرگ‌ها
\usepackage{fancyhdr}
%
\usepackage{setspace}
\usepackage{algorithm}
\usepackage{algorithmic}
\usepackage{subfigure}
\usepackage[subfigure]{tocloft}

% بسته‌ای برای ظاهر شدن «مراجع» و «نمایه» در فهرست مطالب
\usepackage[nottoc]{tocbibind}
% دستورات مربوط به ایجاد نمایه
\usepackage{makeidx}
\makeindex
% رنگهای موردنیاز در کدنویسی
\usepackage[usenames,dvipsnames]{xcolor}
% بسته مورد نیاز برای نوشتن کدهای برنامه نویسی در نوشتار 
\usepackage{listings}
% بسته موردنیاز برای رسم نمودارهای زیبا
\usepackage{tikz}

%%%%%%%%%%%%%%%%%%%%%%%%%%
% فراخوانی بسته زی‌پرشین و تعریف قلم فارسی و انگلیسی
\usepackage{xepersian}
\settextfont[Scale=1.2]{HM XNiloofar}
\setlatintextfont[Scale=0.9]{Times New Roman}

%%%%%%%%%%%%%%%%%%%%%%%%%%
% چنانچه مایلید اعداد فرمولها با قلمی به جز Persian Modern حروف‌چینی شوند، خط زیر را فعال و قلم موردنظر خود را مشخص کنید.
%\setdigitfont[Scale=1]{HM XNiloofar}
%%%%%%%%%%%%%%%%%%%%%%%%%%
% تعریف قلم‌های فارسی و انگلیسی اضافی برای استفاده در بعضی از قسمت‌های متن
\defpersianfont\titlefont[Scale=1]{HM XTitr}
\setiranicfont[Scale=1]{HM XNiloofar Oblique}				% ایرانیک، خوابیده به راست
% \defpersianfont\nastaliq[Scale=1.2]{IranNastaliq}

%%%%%%%%%%%%%%%%%%%%%%%%%%
% دستوری برای حذف کلمه «چکیده»
\renewcommand{\abstractname}{}
% دستوری برای حذف کلمه «abstract»
%\renewcommand{\latinabstract}{}
% دستوری برای تغییر نام کلمه «اثبات» به «برهان»
\renewcommand\proofname{\textbf{برهان}}
% دستوری برای تغییر نام کلمه «کتاب‌نامه» به «مراجع»
\renewcommand{\bibname}{مراجع}
% دستوری برای تعریف واژه‌نامه انگلیسی به فارسی
\newcommand\persiangloss[2]{#1\dotfill\lr{#2}\\}
% دستوری برای تعریف واژه‌نامه فارسی به انگلیسی 
\newcommand\englishgloss[2]{#2\dotfill\lr{#1}\\}
% تعریف دستور جدید «\پ» برای خلاصه‌نویسی جهت نوشتن عبارت «پروژه/پایان‌نامه/رساله»
\newcommand{\پ}{پروژه/پایان‌نامه/رساله }

%\newcommand\BackSlash{\char`\\}

%%%%%%%%%%%%%%%%%%%%%%%%%%
\SepMark{-}

% تعریف و نحوه ظاهر شدن عنوان قضیه‌ها، تعریف‌ها، مثال‌ها و ...
\theoremstyle{definition}
\newtheorem{definition}{تعریف}[section]
\theoremstyle{theorem}
\newtheorem{theorem}[definition]{قضیه}
\newtheorem{lemma}[definition]{لم}
\newtheorem{proposition}[definition]{گزاره}
\newtheorem{corollary}[definition]{نتیجه}
\newtheorem{remark}[definition]{ملاحظه}
\theoremstyle{definition}
\newtheorem{example}[definition]{مثال}

%\renewcommand{\theequation}{\thechapter-\arabic{equation}}
%\def\bibname{مراجع}
\numberwithin{algorithm}{chapter}
\def\listalgorithmname{فهرست الگوریتم‌ها}
\def\listfigurename{فهرست تصاویر}
\def\listtablename{فهرست جداول}

%%%%%%%%%%%%%%%%%%%%%%%%%%%%
% دستورهایی برای سفارشی کردن سربرگ صفحات
% \newcommand{\SetHeader}{
% \csname@twosidetrue\endcsname
% \pagestyle{fancy}
% \fancyhf{} 
% \fancyhead[OL,EL]{\thepage}
% \fancyhead[OR]{\small\rightmark}
% \fancyhead[ER]{\small\leftmark}
% \renewcommand{\chaptermark}[1]{%
% \markboth{\thechapter-\ #1}{}}
% }
%%%%%%%%%%%%5
%\def\MATtextbaseline{1.5}
%\renewcommand{\baselinestretch}{\MATtextbaseline}
\doublespacing
%%%%%%%%%%%%%%%%%%%%%%%%%%%%%
% دستوراتی برای اضافه کردن کلمه «فصل» در فهرست مطالب

\newlength\mylenprt
\newlength\mylenchp
\newlength\mylenapp

\renewcommand\cftpartpresnum{\partname~}
\renewcommand\cftchappresnum{\chaptername~}
\renewcommand\cftchapaftersnum{:}

\settowidth\mylenprt{\cftpartfont\cftpartpresnum\cftpartaftersnum}
\settowidth\mylenchp{\cftchapfont\cftchappresnum\cftchapaftersnum}
\settowidth\mylenapp{\cftchapfont\appendixname~\cftchapaftersnum}
\addtolength\mylenprt{\cftpartnumwidth}
\addtolength\mylenchp{\cftchapnumwidth}
\addtolength\mylenapp{\cftchapnumwidth}

\setlength\cftpartnumwidth{\mylenprt}
\setlength\cftchapnumwidth{\mylenchp}	

\makeatletter
{\def\thebibliography#1{\chapter*{\refname\@mkboth
   {\uppercase{\refname}}{\uppercase{\refname}}}\list
   {[\arabic{enumi}]}{\settowidth\labelwidth{[#1]}
   \rightmargin\labelwidth
   \advance\rightmargin\labelsep
   \advance\rightmargin\bibindent
   \itemindent -\bibindent
   \listparindent \itemindent
   \parsep \z@
   \usecounter{enumi}}
   \def\newblock{}
   \sloppy
   \sfcode`\.=1000\relax}}
   
% اگر مایلید در شماره‌گذاری حرفی و ابجد به جای آ از الف استفاده شود دستورات زیر را فعال کنید.   
% \def\abj@num@i#1{%
  % \ifcase#1\or الف\or ب\or ج\or د%
           % \or ه‍\or و\or ز\or ح\or ط\fi
  % \ifnum#1=\z@\abjad@zero\fi}   
  
  % \def\@harfi#1{\ifcase#1\or الف\or ب\or پ\or ت\or ث\or
% ج\or چ\or ح\or خ\or د\or ذ\or ر\or ز\or ژ\or س\or ش\or ص\or ض\or ط\or ظ\or ع\or غ\or
% ف\or ق\or ک\or گ\or ل\or م\or ن\or و\or ه\or ی\else\@ctrerr\fi}

\makeatother

%%%%%%%%%%%%%%% امکان درج کد در سند
%  در این قسمت تمام ابزارهای مورد نیاز در نوشتن برنامه ها اورده شده 
%  است. با استفاده از این ابزارهای می‌توان برنامه های مورد نیاز را در مستند جای داد.
\lstset{% general command to set parameter(s) 
basicstyle=\small, % print whole listing small
keywordstyle=\color{blue}\bfseries,
% underlined bold black keywords
identifierstyle=, % nothing happens
stringstyle=\ttfamily\color{red},
commentstyle=\color{LimeGreen}, % white comments
stringstyle=\ttfamily\color{red}, % typewriter type for strings
showstringspaces=false} % no special string spaces


\begin{document}
\frontmatter
%%%%%%%%%%%%%%%%%%%%%%%%%%%%%%%%%%%%%%%%%
%صفحه‌ی بسم الله
%\thispagestyle{empty}
%\centerline{{\includegraphics[width=9 cm]{*}}}
%\newpage
%\thispagestyle{empty}
%\clearpage
%~~~
%%%%%%%%%%%%%%%%%%%%%%%%%%%%%%%%%%%%%%%%%
\clearpage
\pagenumbering{adadi}
% در این فایل، عنوان پایان‌نامه، مشخصات خود، متن تقدیمی‌، ستایش، سپاس‌گزاری و چکیده پایان‌نامه را به فارسی، وارد کنید.
% توجه داشته باشید که جدول حاوی مشخصات پایان‌نامه/رساله و همچنین، مشخصات داخل آن، به طور خودکار، درج می‌شود.
%%%%%%%%%%%%%%%%%%%%%%%%%%%%%%%%%%%%
% دانشگاه خود را وارد کنید
\university{شهید بهشتی}
% دانشکده، آموزشکده و یا پژوهشکده  خود را وارد کنید
\faculty{علوم ریاضی}
% گروه آموزشی خود را وارد کنید
\degree {کارشناسی ارشد} 
% گروه آموزشی خود را وارد کنید
\subject{ریاضی }
% گرایش خود را وارد کنید
\field{منطق ریاضی}
% عنوان پایان‌نامه را وارد کنید
\title{منطق شناختی پویای احتمالاتی}
% نام استاد(ان) راهنما را وارد کنید
\firstsupervisor{دکتر مرتضی منیری}
%\secondsupervisor{استاد راهنمای دوم}
% نام استاد(دان) مشاور را وارد کنید. چنانچه استاد مشاور ندارید، دستور پایین را غیرفعال کنید.
%\firstadvisor{استاد مشاور اول}
%\secondadvisor{استاد مشاور دوم}
% نام پژوهشگر را وارد کنید
\name{امیرحسین}
% نام خانوادگی پژوهشگر را وارد کنید
\surname{شرفی}
% تاریخ پایان‌نامه را وارد کنید
\thesisdate{1390}
% کلمات کلیدی پایان‌نامه را وارد کنید
\keywords{منطق شناختی، منطق پویا، منطق احتمالاتی، به‌روزرسانی، \lr{Monty Hall}}
% چکیده پایان‌نامه را وارد کنید
\fa-abstract{\noindent
در این پایان‌نامه ابتدا نمونه‌ای از منطق‌های شناختی احتمالاتی (PEL) را معرفی کرده و تمامیت آن را اثبات می‌کنیم. سپس با در نظر گرفتن مدلی ساده‌ از این منطق، در راستای تعمیم آن به منطق‌های پویا که تغییر اطلاعات در سناریوهای چند عاملی را مدل می‌کنند پیش می‌رویم.
\\
پس از توصیف مختصری از منطق‌های شناختی پویای غیر احتمالاتی، منطق شناختی پویای احتمالاتی (PDEL) را نیز با در نظر گرفتن سه گونه‌ی طبیعی احتمال، یعنی احتمال {\prior} جهان‌ها، احتمال رخداد عمل‌ها بر اساس فرایندی متناظر با دیدگاه عامل‌ها و احتمال خطا در مشاهده‌ی عمل‌ها، معرفی خواهیم کرد. این سه گونه، شیوه‌ی به‌روزرسانی تعمیم‌یافته‌ای در اختیار می‌گذارند که روشی است مناسب و طبیعی برای مدل‌سازی جریان اطلاعات.
\\
سپس برای اینکه تمامیت منطق شناختی پویای احتمالاتی را با استفاده از تمامیت منطق شناختی احتمالاتی اثبات کنیم، اصول موضوعه‌ی صحیحی ارایه می‌کنیم تا فرمول‌های شامل عملگر پویا را به فرمول‌هایی فاقد این عملگر در زبان ایستای متناظر تحویل کنند.
\\
سرانجام گونه‌ای از منطق شناختی پویای احتمالاتی، مورد نیاز برای حل معمای \lr{Monty  Hall}\!\!\! ، ارایه کرده و تمامیت آن را اثبات می‌کنیم. سپس راه‌حلی صوری برای این معما در این منطق بدست می‌آوریم. 
}
\newpage
\thispagestyle{empty}
\vtitle
\newpage
\thispagestyle{empty}
\clearpage
~~~
\newpage
\thispagestyle{empty}
\ \\ \\ \\ \\ \\ \\ \\
{\dav
\begin{center}
كلية حقوق اعم از چاپ و تكثير، نسخه برداري ، ترجمه، اقتباس و ... از اين پايان نامه براي دانشگاه شهيد بهشتي محفوظ است.
 نقل  مطالب با ذكر مأخذ آزاد است.
\end{center}
}
\newpage
\thispagestyle{empty}
\clearpage
~~~
%\newpage
%\thispagestyle{empty}
%\centerline{{\includegraphics[width=20 cm]{replyrecord}}}
%\newpage
%\thispagestyle{empty}
%\clearpage
%~~~
\newpage
 % پایان‌نامه خود را تقدیم کنید!
\begin{acknowledgementpage}

\vspace{4cm}

{\nastaliq
{\Large
 تقدیم به آنکس که بداند و بداند که بداند 
\vspace{1.5cm}

\newdimen\xa
\xa=\textwidth
\advance \xa by -11cm
\hspace{\xa}
و آنکس که نداند و بداند که نداند و بخواهد که بداند
}}
\end{acknowledgementpage}
\newpage
\thispagestyle{empty}
\clearpage
~~~
%%%%%%%%%%%%%%%%%%%%%%%%%%%%%%%%%%%%
\newpage
\thispagestyle{empty}
% ستایش
\baselineskip=.750cm
\ \\ \\
 
{\nastaliq
رسيدن، به دانش است و به كردار نیک...%
}\\
\vspace{.5cm}\\
{\scriptsize\nastaliq
{
 و بی دانش به كردار نيك هم نتوان رسيد، كه نيكى را پيشتر ببايد شناختن، آنگاه بجاى آوردن. پس دانش به همه حال مى‌ببايد تا به رستگارى توان رسيدن. و چون دانش راه آمد، به بهترين چيزها كه آدمى را تواند بودن. و در اوّل آفرينش حاصل نيست و بعضى از آن بى‌رنج و انديشه حاصل شود، پس هرآينه مهمتر چيزى باشد كه در حاصل كردنش عمر گذرانند، ليكن برخى هست كه بى‌انديشه حاصل آيد و بعضى را ناچار به انديشه حاجت بود، و آنچه به انديشه حاصل شود دانسته‌اى خواهد كه درو انديشه كنند تا اين نادانسته بدان انديشه كه در آن دانسته كنند دانسته شود، و از هر دانسته هر نادانسته را نتوان شناخت، بلكه هر نادانسته را به دانسته‌اى كه در خور او بود توان شناخت. و منطق آن علم است كه درو راه انداختن نادانسته به دانسته دانسته شود...
 }}
 
\vspace{.5cm}
{\nastaliq
\newdimen\xb
\xb=\textwidth
\advance \xb by -8.5cm
\hspace{\xb}
پس منطق ناگزير آمد بر جوينده‌ی رستگاری.
\RTLfootnote{مقدمه‌ی رساله‌ی منطق دانشنامه‌ی علائی، شیخ‌الرئیس ابن‌سینا}
}
\newpage
\thispagestyle{empty}
\clearpage
~~~
%%%%%%%%%%%%%%%%%%%%%%%%%%%%%%%%%%%%
\newpage
\thispagestyle{empty}
% سپاس‌گزاری
{\nastaliq
سپاس‌گزاری...
}
\\[2cm]
ستایش و سپاس اولاً و بالذات مخصوص خداوندی است که منطق را فطرتاً در وجود آدمی نهاد.

در آغاز وظیفه‌ی خود می‌دانم از راهنمایی‌ها و زحمات دکتر منیری در به ثمر رسیدن این پایان‌نامه قدردانی نمایم.

همچنین جا دارد که تشکری ویژه داشته باشم از دوست عزیزم علی ولی‌زاده، که با همراهی و مساعدت او به فراگیری منطق‌های شناختی پرداختیم و مقالات مرتبط با پایان‌نامه را به دقت مطالعه کرده و مسائل و مشکلات پیش‌آمده را با نویسندگان مقالات در میان گذاشتیم.

و نیز از دکتر رسول رمضانیان، دکتر کویی و دکتر سَک بواسطه‌ی رفع مشکلاتی چند از پایان‌نامه و تمامی اساتیدی که بر پیشرفت علمی اینجانب تأثیرگذار بودند بالاخص استاد دکتر محمد مهدی ابراهیمی سپاسگزارم.

در آخر لازم است از آقایان وفا خلیقی (پدیدآورنده‌ی \lr{XePersian})، محمود امین‌طوسی، هادی صفی‌اقدم و وحید دامن‌افشان و دیگر دوستانی که در سایت وزین \href{www.parsilatex.com}{\lr{www.parsilatex.com}} به راهنمایی کاربران \lr{\TeX} می‌پردازند قدردانی نمایم.

و بوسه می‌زنم بر دستان پدر و مادر خویش.

% با استفاده از دستور زیر، امضای شما، به طور خودکار، درج می‌شود
\signature 
\newpage
\thispagestyle{empty}
\clearpage
~~~
\newpage
%{\small
\abstractview
%}
\newpage
\thispagestyle{empty}
\clearpage
~~~
\newpage
\chapter*{پیش‌گفتار}
\markboth{پیش‌گفتار}{پیش‌گفتار}

منطق شناختی احتمالاتی پویا منطقی نسبتاً جدید است که قبل از مطالعه‌ی آن می‌بایست منطق شناختی، منطق شناختی پویا و منطق شناختی احتمالاتی معرفی شده باشد و به فراخور در فصول سه‌گانه‌ به معرفی و توصیف هریک خواهیم پرداخت. 

مقدمه را با بهره‌جستن از مقدمه‌ی مقاله‌ی \citep{Kooi2003} نگاشتم و البته هر کجا که لازم بود از مقدمه‌های دیگر مقالات نیز استفاده کردم.

در فصل 1 ابتدا با کمک \citep{DELDitmarsch2007} و \citep{Sack2007} به خلاصه‌ای از منطق شناختی می‌پردازیم، سپس بر مبنای \citep{Fagin1994} منطق شناختی احتمالاتی را که همان منطق شناختی سنتی است به اضافه‌ی توانایی استدلال درباره‌ی احتمال معرفی کرده و تمامیت آن را اثبات می‌کنیم. 

فصل 2 بر اساس \citep{Benthem2009} نوشته شده است و به معرفی منطق‌های پویا اعم از شناختی پویا و شناختی پویای احتمالاتی اختصاص دارد. در این فصل پس از معرفی منطق اعلان عمومی و سپس گونه‌ای تا حدی تعمیم یافته از منطق شناختی پویا با در نظر گرفتن مدلی ساده‌ از منطق‌ شناختی احتمالاتی، منطق شناختی پویای احتمالاتی معرفی شده و تمامیت آن اثبات می‌شود. برای آنکه منطق‌های شناختی پویا را احتمالاتی کنیم ابتدا سه گونه‌ی طبیعی احتمال را تعریف می‌کنیم که عبارتند از: احتمال {\prior} جهان‌ها، احتمال رخداد عمل‌ها بر اساس فرایندی متناظر با دیدگاه عامل‌ها و احتمال خطا در مشاهده‌ی عمل‌ها. برای اثبات تمامیت منطق‌های پویا اصول موضوعه‌ای مطرح می‌شود تا بتوان معادل با هر فرمول در این منطق‌ها، با حذف عملگر پویا، فرمولی در منطق‌های ایستا بدست آورد. این اصول اثرات متقابل عملگر پویا با اتم‌ها و عملگرهای بولی و شناختی را توصیف می‌کنند. پس از اثبات صحت، با کمک آنها تمامیت منطق‌های پویا را از تمامیت منطق‌های ایستا نتیجه می‌گیریم.

هنگامی که به همراه ولی‌زاده به مطالعه و اثبات جزئیات مقاله‌ی \citep{Benthem2009} مشغول بودیم با مسائل و مشکلاتی برخورد کردیم که به تناسب در بخش‌های مختلف فصل ۲ با عنوان ملاحظه به آنها اشاره خواهم کرد.

فصل 3 نیز بر گرفته از \citep{Kooi2003} است و به منظور ارائه‌ی مثالی برای نشان دادن کاربرد منطق شناختی پویای احتمالاتی نگاشته شده است. مثالی که ارائه می‌شود معمای معروف \lr{Monty Hall} است که پیش از این مقاله راه‌حلی صوری برای آن داده نشده بود، ولی ما بر اساس این مقاله به راه‌حلی صوری با کمک گونه‌ای از منطق‌های شناختی پویای احتمالاتی دست می‌یابیم که به زیبایی جوابی معقول در اختیار می‌گذارد. این راه‌حل زیبا را نمی‌توانستم با جزئیات بیان کنم اگر از \citep{Kooithesise2003} استفاده نمی‌کردم و آن را نیافتم مگر با راهنمایی دکتر کویی.

\tableofcontents
%\listoffigures
\clearpage{\pagestyle{empty}\cleardoublepage}
%برای توضیح این دستور به راهنمای بسته‌ی fancyhdr مراجعه شود
\mainmatter
\clearpage
\phantomsection
\addcontentsline{toc}{chapter}{مقدمه}
\chapter*{مقدمه}\markboth{مقدمه}{مقدمه}

در سال ۱۹۵۱ فان رایت\LTRfootnote{\lr{von Wright}} کتابی با عنوان «مقاله‌ای در منطق موجهات» \citep{Wright1951} منتشر کرد. هینتیکا\LTRfootnote{\lr{Hintikka}} با ایده‌هایی که از این مقاله گرفته بود در سال ۱۹۶۲ کتابی با عنوان «دانش و باور، مقدمه‌ای بر منطق مبتنی بر این دو مفهوم» \citep{Hintikka1962} به چاپ رساند. وی در این کتاب به کمک مفهوم جهان‌های ممکن مدلی برای دانش و باور ارائه کرد و به همین دلیل بسیاری او را پدر منطق شناختی می‌دانند. هدف اصلی او واکاوی مفهومی دانش و باور بود ولی پس از او عبارت «شناخت» در محدوده‌ای فراتر به کار گرفته شد، اعم از باور و هر روشی که یک عامل می‌تواند دانشی را بدست آورد. در اواخر دهه‌ی ۱۹۷۰ منطق شناختی مورد توجه دانشمندان فعال در شاخه‌هایی مانند هوش مصنوعی، فلسفه و نظریه بازی‌ها قرار گرفت.
در دهه ۸۰ محققین علوم کامپیوتر به منطق شناختی روی آوردند، فاگین\LTRfootnote{\lr{Fagin}}، هالپرن\LTRfootnote{\lr{Halpern}}، موزِز\LTRfootnote{\lr{Moses}} و وَردی\LTRfootnote{\lr{Vardi}} که از این دسته به حساب می‌آمدند مقالاتی را که در طی حدود ۱۰ سال در مورد منطق شناختی به چاپ رسانده بودند در کتابی با نام «استدلال درباره دانش» \citep{RAKFagin1995} جمع آوری کرده و در سال ۱۹۹۵ منتشر کردند.

منطق شناختی، منطقی وجهی است که به استدلال برمبنای دانش\LTRfootnote{\lr{information}} و فرادانش\LTRfootnote{\lr{higher order information}}\index{فرادانش} می‌پردازد، فرادانش دانشی است که عاملی درباره‌ی دانش خود و یا دانش دیگر عامل‌ها دارد. برای روشن شدن موضوع مثالی را با سه بازیکن 1، 2 و 3، و سه کارت قرمز، سفید وآبی در نظر بگیرید. کارت‌ها در میان بازیکنان به این صورت توزیع شده‌اند که قرمز، سفید و آبی به ترتیب در دستان 1، 2 و 3 قرار دارد. فرض کنید بازیکنان تنها کارت خویش را می‌بینند و همه می‌دانند که کارت‌ها به گونه‌ای میانشان توزبع شده است که هریک فقط یک کارت در دست دارند.  بوسیله‌ی منطق شناختی می‌توان جملات پیچیده‌ای را مانند «بازیکن 1 می‌داند که بازیکن 2 نمی‌داند که چه کارتی در دستان بازیکن 3 است.» صوری کرد. منطق شناخت با چنین فرادانش‌هایی سر و کار دارد. حتی فرادانش‌های پیچیده‌تری مانند همه‌دانی مشترک وجود دارد. برای نمونه در مثال بالا علم به اینکه دقیقاً سه کارت وجود دارد و علم به رنگ کارت‌ها همه‌دانی مشترک است. منطق شناختی همچنین واکاوی مفهومی خوبی از همه‌دانی مشترک در اختیار می‌گذارد.

اگرچه منطق شناختی آنالیز مناسبی از فرادانش در اختیار می‌گذارد ولی بررسی تغییر دانش خارج از گستره‌ی این منطق است. منطق‌های شناختی پویا منطق‌ شناختی را به گونه‌ای توسیع می‌دهند که استدلال درباره‌ی تغییرات دانش نیز امکان‌پذیر باشد. این توسیع از طرفی از نوعی معناشناسی زبان طبیعی الهام گرفته شد که در آن معنای جمله بعنوان طریقی برای تغییر داده‌های کسانی که آن را می‌شنوند در نظر گرفته می‌شود، و از طرف دیگر از مطالعه‌ی بازی‌ها که تغییر داده‌ها و فرادانش‌ها نقش بسزایی در آنها ایفا می‌کند. سیستم‌های منطقی مختلفی بر این اساس در طول سال‌ها شکل گرفته است که برجسته‌ترین آنها عبارتند از \citep{Gerbrandy1997} (الهام گرفته از \citep{Veltman1996})، \citep{Batlag1998} و  \citep{Cate2002}.

در منطق شناختی پویا تغییر وضعیت موجود توسط داده‌های جدید را به‌روزرسانی می‌خوانیم. ساده‌ترین مثال زمانی است که عاملی می‌فهمد که گزاره‌ی $ \varphi $ برقرار است. به‌روزرسانی با یک گزاره در این مثال به این معنی است که گزینه‌هایی که عامل ممکن می‌دانست ولی در آنها گزاره‌ی $ \varphi $ برقرار نیست حذف می‌شوند. در یک سیستم چند عاملی ممکن است عامل‌های مختلف دسترسی مختلفی به داده‌های جدید داشته باشند و همچنین اطلاعات عامل‌ها درباره‌ی دیگر عامل‌ها نقش بازی کند، از این رو می‌توان به‌روزرسانی پیچیده‌تری را در مثال قبل مدل کرد: فرض کنید بازیکن 1 کارت خود را به بازیکن 2 نشان دهد و بازیکن 3 نیز این را ببیند ولی از محتوای کارت خبردار نشود. در نتیجه دانش بازیکنان به این صورت تغییر می‌کند: بازیکن 2 می‌داند که محتوای کارت بازیکن 1 چیست، بازیکن 3 می‌داند که بازیکن 2 می‌داند محتوای کارت بازیکن 1 جیست ولی خودش محتوا را نمی‌داند و بازیکن 1 می‌داند که بازیکن 3 این را می‌داند.

اگرچه نظریه‌ی احتمال منطق نیست لکن حوزه‌ی مطالعاتی مناسبی برای منطق است، زیرا در بسیاری از حوزه‌های کاربردی به منظور استدلال درباره‌ی دانش، اهمیت توانایی استدلال درباره‌ی احتمال رخدادهای معین به همراه دانش عامل‌ها رخ می‌نماید و اغلب احتمال به عنوان نظریه‌ای برای مدلسازی استدلال مطرح می‌شود.
از این رو همه‌ی مقالات منتشر شده در علم اقتصاد که به استدلال درباره‌ی دانش می‌پردازند (که بازگشت می‌کنند به مقاله‌ی اصلی اومان \citep{Aumann1976}) با ساختاری احتمالاتی مدل می‌شوند، هرچند آنها زبانی منطقی که بصورتی روشن استدلال درباره‌ی احتمال را جایز کند در نظر نگرفته‌اند. با این اوصاف تلاش‌هایی در جهت بیرون کشیدن منطق بعنوان بهترین راه استدلال از دل نظریه‌ی احتمال صورت گرفته است. که یکی از مناسب‌تزین آنها منطق احتمالاتی است که در \citep{Fagin1990} معرفی شده است.
\clearpage{\pagestyle{empty}\cleardoublepage}
\chapter{راهنمای استفاده از کلاس}
\thispagestyle{empty}
\section{مقدمه}
حروف‌چینی پروژه کارشناسی، پایان‌نامه یا رساله یکی از موارد پرکاربرد استفاده از زی‌پرشین است. از طرفی، یک پروژه، پایان‌نامه یا رساله،  احتیاج به تنظیمات زیادی از نظر صفحه‌آرایی  دارد که ممکن است برای
یک کاربر مبتدی، مشکل باشد. به همین خاطر، برای راحتی کار کاربر، کلاس حاضر با نام 
 \LRE{\verb!tabriz-thesis!}
 برای حروف‌چینی پروژه‌ها، پایان‌نامه‌ها و رساله‌های دانشگاه تبریز با استفاده از نرم‌افزار زی‌پرشین،  آماده شده است. این فایل به 
گونه‌ای طراحی شده است که کلیه خواسته‌های مورد نیاز  مدیریت تحصیلات تکمیلی دانشگاه تبریز را برآورده می‌کند و نیز، حروف‌چینی بسیاری
از قسمت‌های آن، به طور خودکار انجام می‌شود.

کلیه فایل‌های لازم برای حروف‌چینی با کلاس گفته شده، داخل پوشه‌ای به نام
 \LRE{\verb!tabriz-thesis!}
  قرار داده شده است. توجه داشته باشید که برای استفاده از این کلاس باید فونت‌های
\LRE{\verb!XB Niloofar!}،
 \verb!Yas!
 و
  \verb!IranNastaliq!
    روی سیستم شما نصب شده باشد.
\section{این همه فایل؟!}\label{sec2}
از آنجایی که یک پایان‌نامه یا رساله، یک نوشته بلند محسوب می‌شود، لذا اگر همه تنظیمات و مطالب پایان‌نامه را داخل یک فایل قرار بدهیم، باعث شلوغی
و سردرگمی می‌شود. به همین خاطر، قسمت‌های مختلف پایان‌نامه یا رساله  داخل فایل‌های جداگانه قرار گرفته است. مثلاً تنظیمات پایه‌ای کلاس، داخل فایل
\LRE{\verb!tabriz-thesis.cls!}، 
تنظیمات قابل تغییر توسط کاربر، داخل 
\verb!commands.tex!،
قسمت مشخصات فارسی پایان‌نامه، داخل 
\LRE{\verb!fa-title.tex!}،
مطالب فصل اول، داخل 
\verb!chapter1!
و ... قرار داده شده است. نکته مهمی که در اینجا وجود دارد این است که از بین این  فایل‌ها، فقط فایل 
\LRE{\verb!tabriz-thesis.tex!}
قابل اجرا است. یعنی بعد از تغییر فایل‌های دیگر، برای دیدن نتیجه تغییرات، باید این فایل را اجرا کرد. بقیه فایل‌ها به این فایل، کمک می‌کنند تا بتوانیم خروجی کار را ببینیم. اگر به فایل 
\LRE{\verb!tabriz-thesis.tex!}
دقت کنید، متوجه می‌شوید که قسمت‌های مختلف پایان‌نامه، توسط دستورهایی مانند 
\verb!input!
و
\verb!include!
به فایل اصلی، یعنی 
\LRE{\verb!tabriz-thesis.tex!}
معرفی شده‌اند. بنابراین، فایلی که همیشه با آن سروکار داریم، فایل 
\LRE{\verb!tabriz-thesis.tex!}
است.
در این فایل، فرض شده است که پایان‌نامه یا رساله، از ۳ فصل و یک پیوست، تشکیل شده است. با این حال، اگر
  پایان‌نامه یا رساله، بیشتر از ۳ فصل و یک پیوست است، باید خودتان فصل‌های بیشتر را به این فایل، اضافه کنید. این کار، بسیار ساده است. فرض کنید بخواهید یک فصل دیگر هم به پایان‌نامه، اضافه کنید. برای این کار، کافی است یک فایل با نام 
\verb!chapter4!
و با پسوند 
\verb!.tex!
بسازید و آن را داخل پوشه 
\LRE{\verb!tabriz-thesis!}
قرار دهید و سپس این فایل را با دستور 
\verb!\include{chapter4}!
داخل فایل
\LRE{\verb!tabriz-thesis.tex!}
و بعد از دستور
\verb!\chapter{اندازه‌ها و ارزیابی‌ها}
\thispagestyle{empty}
\section{اندازه‌ها و تابعی‌های خطی مثبت روی $\mathrm{C(X)}$}
فرض کنید $X$ یک فضای توپولوژیکی روی ...
\section{تابعی‌های خطی}
در این بخش ...!
 قرار دهید.
\section{از کجا شروع کنم؟}
قبل از هر چیز، بدیهی است که باید یک توزیع تِک مناسب مانند 
\verb!Live TeX!
و یک ویرایش‌گر تِک مانند
\verb!Texmaker!
را روی سیستم خود نصب کنید.  نسخه بهینه شده \verb!Texmaker!  را می‌توانید  از سایت 
 \href{http://www.parsilatex.com}{پارسی‌لاتک}%
\LTRfootnote{\url{http://www.parsilatex.com}}
 و \verb!Live TeX!  را هم می‌توانید از 
 \href{http://www.tug.org/texlive}{سایت رسمی آن}%
\LTRfootnote{\url{http://www.tug.org/texlive}}
 دانلود کنید.
 
در مرحله بعد، سعی کنید که  یک پشتیبان از پوشه 
\LRE{\verb!tabriz-thesis!}
 بگیرید و آن را در یک جایی از هارددیسک سیستم خود ذخیره کنید تا در صورت خراب کردن فایل‌هایی که در حال حاضر، با آن‌ها کار می‌کنید، همه چیز را از 
 دست ندهید.
 
 حال اگر نوشتن \پ اولین تجربه شما از کار با لاتک است، توصیه می‌شود که یک‌بار به طور سرسری، کتاب «%
\href{http://www.tug.ctan.org/tex-archive/info/lshort/persian/lshort.pdf}{مقدمه‌ای نه چندان کوتاه بر
\lr{\LaTeXe}}\LTRfootnote{\url{http://www.tug.ctan.org/tex-archive/info/lshort/persian/lshort.pdf}}»
   ترجمه دکتر مهدی امیدعلی، عضو هیات علمی دانشگاه شاهد را مطالعه کنید. این کتاب، کتاب بسیار کاملی است که خیلی از نیازهای شما در ارتباط با حروف‌چینی را برطرف می‌کند.
 
 
بعد از موارد گفته شده، فایل 
\LRE{\verb!tabriz-thesis.tex!}
و
\LRE{\verb!fa-title!}
را باز کنید و مشخصات پایان‌نامه خود مثل نام، نام خانوادگی، عنوان پایان‌نامه و ... را جایگزین مشخصات موجود در فایل
\LRE{\verb!fa-title!}
 کنید. دقت داشته باشید که نیازی نیست 
نگران چینش این مشخصات در فایل پی‌دی‌اف خروجی باشید. فایل 
\LRE{\verb!tabriz-thesis.cls!}
همه این کارها را به طور خودکار برای شما انجام می‌دهد. در ضمن، موقع تغییر دادن دستورهای داخل فایل
\LRE{\verb!fa-title!}
 کاملاً دقت کنید. این دستورها، خیلی حساس هستند و ممکن است با یک تغییر کوچک، موقع اجرا، خطا بگیرید. برای دیدن خروجی کار، فایل 
\LRE{\verb!fa-title!}
 را 
\verb!Save!، 
(نه 
\verb!As Save!)
کنید و بعد به فایل 
\LRE{\verb!tabriz-thesis.tex!}
برگشته و آن را اجرا کنید. حال اگر می‌خواهید مشخصات انگلیسی \پ را هم عوض کنید، فایل 
\LRE{\verb!en-title!}
را باز کنید و مشخصات داخل آن را تغییر دهید.%
\RTLfootnote{
برای نوشتن پروژه کارشناسی، نیازی به وارد کردن مشخصات انگلیسی پروژه نیست. بنابراین، این مشخصات، به طور خودکار،
نادیده گرفته می‌شود.
}
 در اینجا هم برای دیدن خروجی، باید این فایل را 
\verb!Save!
کرده و بعد به فایل 
\LRE{\verb!tabriz-thesis.tex!}
برگشته و آن را اجرا کرد.

برای راحتی بیشتر، 
فایل 
\LRE{\verb!tabriz-thesis.cls!}
طوری طراحی شده است که کافی است فقط  یک‌بار مشخصات \پ  را وارد کنید. هر جای دیگر که لازم به درج این مشخصات باشد، این مشخصات به طور خودکار درج می‌شود. با این حال، اگر مایل بودید، می‌توانید تنظیمات موجود را تغییر دهید. توجه داشته باشید که اگر کاربر مبتدی هستید و یا با ساختار فایل‌های  
\verb!cls!
 آشنایی ندارید، به هیچ وجه به این فایل، یعنی فایل 
\LRE{\verb!tabriz-thesis.cls!}
دست نزنید.

نکته دیگری که باید به آن توجه کنید این است که در فایل 
\LRE{\verb!tabriz-thesis.cls!}،
سه گزینه به نام‌های
\verb!bsc!،
\verb!msc!
و
\verb!phd!
برای تایپ پروژه، پایان‌نامه و رساله،
طراحی شده است. بنابراین اگر قصد تایپ پروژه کارشناسی، پایان‌نامه یا رساله را دارید، 
 در فایل 
\LRE{\verb!tabriz-thesis.tex!}
باید به ترتیب از گزینه‌های
\verb!bsc!،
\verb!msc!
و
\verb!phd!
استفاده کنید. با انتخاب هر کدام از این گزینه‌ها، تنظیمات مربوط به آنها به طور خودکار، اعمل می‌شود.    
\section{مطالب \پ را چطور بنویسم؟}
\subsection{نوشتن فصل‌ها}
همان‌طور که در بخش \ref{sec2} گفته شد، برای جلوگیری از شلوغی و سردرگمی کاربر در هنگام حروف‌چینی، قسمت‌های مختلف \پ از جمله فصل‌ها، در فایل‌های جداگانه‌ای قرار داده شده‌اند. 
بنابراین، اگر می‌خواهید مثلاً مطالب فصل ۱ را تایپ کنید، باید فایل‌های 
\LRE{\verb!tabriz-thesis.tex!}
و
\verb!chapter1!
را باز کنید و محتویات داخل فایل 
\verb!chapter1!
را پاک کرده و مطالب خود را تایپ کنید. توجه کنید که همان‌طور که قبلاً هم گفته شد، تنها فایل قابل اجرا، فایل 
\LRE{\verb!tabriz-thesis.tex!}
است. لذا برای دیدن حاصل (خروجی) فایل خود، باید فایل  
\verb!chapter1!
را 
\verb!Save!
کرده و سپس فایل 
\LRE{\verb!tabriz-thesis.tex!}
را اجرا کنید. یک نکته بدیهی که در اینجا وجود دارد، این است که لازم نیست که فصل‌های \پ را به ترتیب تایپ کنید. می‌توانید ابتدا مطالب فصل ۳ را تایپ کنید و سپس مطالب فصل ۱ را تایپ کنید. 

نکته بسیار مهمی که در اینجا باید گفته شود این است که سیستم \lr{\TeX}، محتویات یک فایل تِک را به ترتیب پردازش می‌کند. به عنوان مثال، اگه فایلی، دارای ۴ خط دستور باشد، ابتدا خط ۱، بعد خط ۲، بعد خط ۳ و در آخر، خط ۴ پردازش می‌شود. بنابراین، اگر مثلاً مشغول تایپ مطالب فصل ۳ هستید، بهتر است
که دو دستور 
\verb!\chapter{راهنمای استفاده از کلاس}
\thispagestyle{empty}
\section{مقدمه}
حروف‌چینی پروژه کارشناسی، پایان‌نامه یا رساله یکی از موارد پرکاربرد استفاده از زی‌پرشین است. از طرفی، یک پروژه، پایان‌نامه یا رساله،  احتیاج به تنظیمات زیادی از نظر صفحه‌آرایی  دارد که ممکن است برای
یک کاربر مبتدی، مشکل باشد. به همین خاطر، برای راحتی کار کاربر، کلاس حاضر با نام 
 \LRE{\verb!tabriz-thesis!}
 برای حروف‌چینی پروژه‌ها، پایان‌نامه‌ها و رساله‌های دانشگاه تبریز با استفاده از نرم‌افزار زی‌پرشین،  آماده شده است. این فایل به 
گونه‌ای طراحی شده است که کلیه خواسته‌های مورد نیاز  مدیریت تحصیلات تکمیلی دانشگاه تبریز را برآورده می‌کند و نیز، حروف‌چینی بسیاری
از قسمت‌های آن، به طور خودکار انجام می‌شود.

کلیه فایل‌های لازم برای حروف‌چینی با کلاس گفته شده، داخل پوشه‌ای به نام
 \LRE{\verb!tabriz-thesis!}
  قرار داده شده است. توجه داشته باشید که برای استفاده از این کلاس باید فونت‌های
\LRE{\verb!XB Niloofar!}،
 \verb!Yas!
 و
  \verb!IranNastaliq!
    روی سیستم شما نصب شده باشد.
\section{این همه فایل؟!}\label{sec2}
از آنجایی که یک پایان‌نامه یا رساله، یک نوشته بلند محسوب می‌شود، لذا اگر همه تنظیمات و مطالب پایان‌نامه را داخل یک فایل قرار بدهیم، باعث شلوغی
و سردرگمی می‌شود. به همین خاطر، قسمت‌های مختلف پایان‌نامه یا رساله  داخل فایل‌های جداگانه قرار گرفته است. مثلاً تنظیمات پایه‌ای کلاس، داخل فایل
\LRE{\verb!tabriz-thesis.cls!}، 
تنظیمات قابل تغییر توسط کاربر، داخل 
\verb!commands.tex!،
قسمت مشخصات فارسی پایان‌نامه، داخل 
\LRE{\verb!fa-title.tex!}،
مطالب فصل اول، داخل 
\verb!chapter1!
و ... قرار داده شده است. نکته مهمی که در اینجا وجود دارد این است که از بین این  فایل‌ها، فقط فایل 
\LRE{\verb!tabriz-thesis.tex!}
قابل اجرا است. یعنی بعد از تغییر فایل‌های دیگر، برای دیدن نتیجه تغییرات، باید این فایل را اجرا کرد. بقیه فایل‌ها به این فایل، کمک می‌کنند تا بتوانیم خروجی کار را ببینیم. اگر به فایل 
\LRE{\verb!tabriz-thesis.tex!}
دقت کنید، متوجه می‌شوید که قسمت‌های مختلف پایان‌نامه، توسط دستورهایی مانند 
\verb!input!
و
\verb!include!
به فایل اصلی، یعنی 
\LRE{\verb!tabriz-thesis.tex!}
معرفی شده‌اند. بنابراین، فایلی که همیشه با آن سروکار داریم، فایل 
\LRE{\verb!tabriz-thesis.tex!}
است.
در این فایل، فرض شده است که پایان‌نامه یا رساله، از ۳ فصل و یک پیوست، تشکیل شده است. با این حال، اگر
  پایان‌نامه یا رساله، بیشتر از ۳ فصل و یک پیوست است، باید خودتان فصل‌های بیشتر را به این فایل، اضافه کنید. این کار، بسیار ساده است. فرض کنید بخواهید یک فصل دیگر هم به پایان‌نامه، اضافه کنید. برای این کار، کافی است یک فایل با نام 
\verb!chapter4!
و با پسوند 
\verb!.tex!
بسازید و آن را داخل پوشه 
\LRE{\verb!tabriz-thesis!}
قرار دهید و سپس این فایل را با دستور 
\verb!\include{chapter4}!
داخل فایل
\LRE{\verb!tabriz-thesis.tex!}
و بعد از دستور
\verb!\chapter{اندازه‌ها و ارزیابی‌ها}
\thispagestyle{empty}
\section{اندازه‌ها و تابعی‌های خطی مثبت روی $\mathrm{C(X)}$}
فرض کنید $X$ یک فضای توپولوژیکی روی ...
\section{تابعی‌های خطی}
در این بخش ...!
 قرار دهید.
\section{از کجا شروع کنم؟}
قبل از هر چیز، بدیهی است که باید یک توزیع تِک مناسب مانند 
\verb!Live TeX!
و یک ویرایش‌گر تِک مانند
\verb!Texmaker!
را روی سیستم خود نصب کنید.  نسخه بهینه شده \verb!Texmaker!  را می‌توانید  از سایت 
 \href{http://www.parsilatex.com}{پارسی‌لاتک}%
\LTRfootnote{\url{http://www.parsilatex.com}}
 و \verb!Live TeX!  را هم می‌توانید از 
 \href{http://www.tug.org/texlive}{سایت رسمی آن}%
\LTRfootnote{\url{http://www.tug.org/texlive}}
 دانلود کنید.
 
در مرحله بعد، سعی کنید که  یک پشتیبان از پوشه 
\LRE{\verb!tabriz-thesis!}
 بگیرید و آن را در یک جایی از هارددیسک سیستم خود ذخیره کنید تا در صورت خراب کردن فایل‌هایی که در حال حاضر، با آن‌ها کار می‌کنید، همه چیز را از 
 دست ندهید.
 
 حال اگر نوشتن \پ اولین تجربه شما از کار با لاتک است، توصیه می‌شود که یک‌بار به طور سرسری، کتاب «%
\href{http://www.tug.ctan.org/tex-archive/info/lshort/persian/lshort.pdf}{مقدمه‌ای نه چندان کوتاه بر
\lr{\LaTeXe}}\LTRfootnote{\url{http://www.tug.ctan.org/tex-archive/info/lshort/persian/lshort.pdf}}»
   ترجمه دکتر مهدی امیدعلی، عضو هیات علمی دانشگاه شاهد را مطالعه کنید. این کتاب، کتاب بسیار کاملی است که خیلی از نیازهای شما در ارتباط با حروف‌چینی را برطرف می‌کند.
 
 
بعد از موارد گفته شده، فایل 
\LRE{\verb!tabriz-thesis.tex!}
و
\LRE{\verb!fa-title!}
را باز کنید و مشخصات پایان‌نامه خود مثل نام، نام خانوادگی، عنوان پایان‌نامه و ... را جایگزین مشخصات موجود در فایل
\LRE{\verb!fa-title!}
 کنید. دقت داشته باشید که نیازی نیست 
نگران چینش این مشخصات در فایل پی‌دی‌اف خروجی باشید. فایل 
\LRE{\verb!tabriz-thesis.cls!}
همه این کارها را به طور خودکار برای شما انجام می‌دهد. در ضمن، موقع تغییر دادن دستورهای داخل فایل
\LRE{\verb!fa-title!}
 کاملاً دقت کنید. این دستورها، خیلی حساس هستند و ممکن است با یک تغییر کوچک، موقع اجرا، خطا بگیرید. برای دیدن خروجی کار، فایل 
\LRE{\verb!fa-title!}
 را 
\verb!Save!، 
(نه 
\verb!As Save!)
کنید و بعد به فایل 
\LRE{\verb!tabriz-thesis.tex!}
برگشته و آن را اجرا کنید. حال اگر می‌خواهید مشخصات انگلیسی \پ را هم عوض کنید، فایل 
\LRE{\verb!en-title!}
را باز کنید و مشخصات داخل آن را تغییر دهید.%
\RTLfootnote{
برای نوشتن پروژه کارشناسی، نیازی به وارد کردن مشخصات انگلیسی پروژه نیست. بنابراین، این مشخصات، به طور خودکار،
نادیده گرفته می‌شود.
}
 در اینجا هم برای دیدن خروجی، باید این فایل را 
\verb!Save!
کرده و بعد به فایل 
\LRE{\verb!tabriz-thesis.tex!}
برگشته و آن را اجرا کرد.

برای راحتی بیشتر، 
فایل 
\LRE{\verb!tabriz-thesis.cls!}
طوری طراحی شده است که کافی است فقط  یک‌بار مشخصات \پ  را وارد کنید. هر جای دیگر که لازم به درج این مشخصات باشد، این مشخصات به طور خودکار درج می‌شود. با این حال، اگر مایل بودید، می‌توانید تنظیمات موجود را تغییر دهید. توجه داشته باشید که اگر کاربر مبتدی هستید و یا با ساختار فایل‌های  
\verb!cls!
 آشنایی ندارید، به هیچ وجه به این فایل، یعنی فایل 
\LRE{\verb!tabriz-thesis.cls!}
دست نزنید.

نکته دیگری که باید به آن توجه کنید این است که در فایل 
\LRE{\verb!tabriz-thesis.cls!}،
سه گزینه به نام‌های
\verb!bsc!،
\verb!msc!
و
\verb!phd!
برای تایپ پروژه، پایان‌نامه و رساله،
طراحی شده است. بنابراین اگر قصد تایپ پروژه کارشناسی، پایان‌نامه یا رساله را دارید، 
 در فایل 
\LRE{\verb!tabriz-thesis.tex!}
باید به ترتیب از گزینه‌های
\verb!bsc!،
\verb!msc!
و
\verb!phd!
استفاده کنید. با انتخاب هر کدام از این گزینه‌ها، تنظیمات مربوط به آنها به طور خودکار، اعمل می‌شود.    
\section{مطالب \پ را چطور بنویسم؟}
\subsection{نوشتن فصل‌ها}
همان‌طور که در بخش \ref{sec2} گفته شد، برای جلوگیری از شلوغی و سردرگمی کاربر در هنگام حروف‌چینی، قسمت‌های مختلف \پ از جمله فصل‌ها، در فایل‌های جداگانه‌ای قرار داده شده‌اند. 
بنابراین، اگر می‌خواهید مثلاً مطالب فصل ۱ را تایپ کنید، باید فایل‌های 
\LRE{\verb!tabriz-thesis.tex!}
و
\verb!chapter1!
را باز کنید و محتویات داخل فایل 
\verb!chapter1!
را پاک کرده و مطالب خود را تایپ کنید. توجه کنید که همان‌طور که قبلاً هم گفته شد، تنها فایل قابل اجرا، فایل 
\LRE{\verb!tabriz-thesis.tex!}
است. لذا برای دیدن حاصل (خروجی) فایل خود، باید فایل  
\verb!chapter1!
را 
\verb!Save!
کرده و سپس فایل 
\LRE{\verb!tabriz-thesis.tex!}
را اجرا کنید. یک نکته بدیهی که در اینجا وجود دارد، این است که لازم نیست که فصل‌های \پ را به ترتیب تایپ کنید. می‌توانید ابتدا مطالب فصل ۳ را تایپ کنید و سپس مطالب فصل ۱ را تایپ کنید. 

نکته بسیار مهمی که در اینجا باید گفته شود این است که سیستم \lr{\TeX}، محتویات یک فایل تِک را به ترتیب پردازش می‌کند. به عنوان مثال، اگه فایلی، دارای ۴ خط دستور باشد، ابتدا خط ۱، بعد خط ۲، بعد خط ۳ و در آخر، خط ۴ پردازش می‌شود. بنابراین، اگر مثلاً مشغول تایپ مطالب فصل ۳ هستید، بهتر است
که دو دستور 
\verb!\chapter{راهنمای استفاده از کلاس}
\thispagestyle{empty}
\section{مقدمه}
حروف‌چینی پروژه کارشناسی، پایان‌نامه یا رساله یکی از موارد پرکاربرد استفاده از زی‌پرشین است. از طرفی، یک پروژه، پایان‌نامه یا رساله،  احتیاج به تنظیمات زیادی از نظر صفحه‌آرایی  دارد که ممکن است برای
یک کاربر مبتدی، مشکل باشد. به همین خاطر، برای راحتی کار کاربر، کلاس حاضر با نام 
 \LRE{\verb!tabriz-thesis!}
 برای حروف‌چینی پروژه‌ها، پایان‌نامه‌ها و رساله‌های دانشگاه تبریز با استفاده از نرم‌افزار زی‌پرشین،  آماده شده است. این فایل به 
گونه‌ای طراحی شده است که کلیه خواسته‌های مورد نیاز  مدیریت تحصیلات تکمیلی دانشگاه تبریز را برآورده می‌کند و نیز، حروف‌چینی بسیاری
از قسمت‌های آن، به طور خودکار انجام می‌شود.

کلیه فایل‌های لازم برای حروف‌چینی با کلاس گفته شده، داخل پوشه‌ای به نام
 \LRE{\verb!tabriz-thesis!}
  قرار داده شده است. توجه داشته باشید که برای استفاده از این کلاس باید فونت‌های
\LRE{\verb!XB Niloofar!}،
 \verb!Yas!
 و
  \verb!IranNastaliq!
    روی سیستم شما نصب شده باشد.
\section{این همه فایل؟!}\label{sec2}
از آنجایی که یک پایان‌نامه یا رساله، یک نوشته بلند محسوب می‌شود، لذا اگر همه تنظیمات و مطالب پایان‌نامه را داخل یک فایل قرار بدهیم، باعث شلوغی
و سردرگمی می‌شود. به همین خاطر، قسمت‌های مختلف پایان‌نامه یا رساله  داخل فایل‌های جداگانه قرار گرفته است. مثلاً تنظیمات پایه‌ای کلاس، داخل فایل
\LRE{\verb!tabriz-thesis.cls!}، 
تنظیمات قابل تغییر توسط کاربر، داخل 
\verb!commands.tex!،
قسمت مشخصات فارسی پایان‌نامه، داخل 
\LRE{\verb!fa-title.tex!}،
مطالب فصل اول، داخل 
\verb!chapter1!
و ... قرار داده شده است. نکته مهمی که در اینجا وجود دارد این است که از بین این  فایل‌ها، فقط فایل 
\LRE{\verb!tabriz-thesis.tex!}
قابل اجرا است. یعنی بعد از تغییر فایل‌های دیگر، برای دیدن نتیجه تغییرات، باید این فایل را اجرا کرد. بقیه فایل‌ها به این فایل، کمک می‌کنند تا بتوانیم خروجی کار را ببینیم. اگر به فایل 
\LRE{\verb!tabriz-thesis.tex!}
دقت کنید، متوجه می‌شوید که قسمت‌های مختلف پایان‌نامه، توسط دستورهایی مانند 
\verb!input!
و
\verb!include!
به فایل اصلی، یعنی 
\LRE{\verb!tabriz-thesis.tex!}
معرفی شده‌اند. بنابراین، فایلی که همیشه با آن سروکار داریم، فایل 
\LRE{\verb!tabriz-thesis.tex!}
است.
در این فایل، فرض شده است که پایان‌نامه یا رساله، از ۳ فصل و یک پیوست، تشکیل شده است. با این حال، اگر
  پایان‌نامه یا رساله، بیشتر از ۳ فصل و یک پیوست است، باید خودتان فصل‌های بیشتر را به این فایل، اضافه کنید. این کار، بسیار ساده است. فرض کنید بخواهید یک فصل دیگر هم به پایان‌نامه، اضافه کنید. برای این کار، کافی است یک فایل با نام 
\verb!chapter4!
و با پسوند 
\verb!.tex!
بسازید و آن را داخل پوشه 
\LRE{\verb!tabriz-thesis!}
قرار دهید و سپس این فایل را با دستور 
\verb!\include{chapter4}!
داخل فایل
\LRE{\verb!tabriz-thesis.tex!}
و بعد از دستور
\verb!\include{chapter3}!
 قرار دهید.
\section{از کجا شروع کنم؟}
قبل از هر چیز، بدیهی است که باید یک توزیع تِک مناسب مانند 
\verb!Live TeX!
و یک ویرایش‌گر تِک مانند
\verb!Texmaker!
را روی سیستم خود نصب کنید.  نسخه بهینه شده \verb!Texmaker!  را می‌توانید  از سایت 
 \href{http://www.parsilatex.com}{پارسی‌لاتک}%
\LTRfootnote{\url{http://www.parsilatex.com}}
 و \verb!Live TeX!  را هم می‌توانید از 
 \href{http://www.tug.org/texlive}{سایت رسمی آن}%
\LTRfootnote{\url{http://www.tug.org/texlive}}
 دانلود کنید.
 
در مرحله بعد، سعی کنید که  یک پشتیبان از پوشه 
\LRE{\verb!tabriz-thesis!}
 بگیرید و آن را در یک جایی از هارددیسک سیستم خود ذخیره کنید تا در صورت خراب کردن فایل‌هایی که در حال حاضر، با آن‌ها کار می‌کنید، همه چیز را از 
 دست ندهید.
 
 حال اگر نوشتن \پ اولین تجربه شما از کار با لاتک است، توصیه می‌شود که یک‌بار به طور سرسری، کتاب «%
\href{http://www.tug.ctan.org/tex-archive/info/lshort/persian/lshort.pdf}{مقدمه‌ای نه چندان کوتاه بر
\lr{\LaTeXe}}\LTRfootnote{\url{http://www.tug.ctan.org/tex-archive/info/lshort/persian/lshort.pdf}}»
   ترجمه دکتر مهدی امیدعلی، عضو هیات علمی دانشگاه شاهد را مطالعه کنید. این کتاب، کتاب بسیار کاملی است که خیلی از نیازهای شما در ارتباط با حروف‌چینی را برطرف می‌کند.
 
 
بعد از موارد گفته شده، فایل 
\LRE{\verb!tabriz-thesis.tex!}
و
\LRE{\verb!fa-title!}
را باز کنید و مشخصات پایان‌نامه خود مثل نام، نام خانوادگی، عنوان پایان‌نامه و ... را جایگزین مشخصات موجود در فایل
\LRE{\verb!fa-title!}
 کنید. دقت داشته باشید که نیازی نیست 
نگران چینش این مشخصات در فایل پی‌دی‌اف خروجی باشید. فایل 
\LRE{\verb!tabriz-thesis.cls!}
همه این کارها را به طور خودکار برای شما انجام می‌دهد. در ضمن، موقع تغییر دادن دستورهای داخل فایل
\LRE{\verb!fa-title!}
 کاملاً دقت کنید. این دستورها، خیلی حساس هستند و ممکن است با یک تغییر کوچک، موقع اجرا، خطا بگیرید. برای دیدن خروجی کار، فایل 
\LRE{\verb!fa-title!}
 را 
\verb!Save!، 
(نه 
\verb!As Save!)
کنید و بعد به فایل 
\LRE{\verb!tabriz-thesis.tex!}
برگشته و آن را اجرا کنید. حال اگر می‌خواهید مشخصات انگلیسی \پ را هم عوض کنید، فایل 
\LRE{\verb!en-title!}
را باز کنید و مشخصات داخل آن را تغییر دهید.%
\RTLfootnote{
برای نوشتن پروژه کارشناسی، نیازی به وارد کردن مشخصات انگلیسی پروژه نیست. بنابراین، این مشخصات، به طور خودکار،
نادیده گرفته می‌شود.
}
 در اینجا هم برای دیدن خروجی، باید این فایل را 
\verb!Save!
کرده و بعد به فایل 
\LRE{\verb!tabriz-thesis.tex!}
برگشته و آن را اجرا کرد.

برای راحتی بیشتر، 
فایل 
\LRE{\verb!tabriz-thesis.cls!}
طوری طراحی شده است که کافی است فقط  یک‌بار مشخصات \پ  را وارد کنید. هر جای دیگر که لازم به درج این مشخصات باشد، این مشخصات به طور خودکار درج می‌شود. با این حال، اگر مایل بودید، می‌توانید تنظیمات موجود را تغییر دهید. توجه داشته باشید که اگر کاربر مبتدی هستید و یا با ساختار فایل‌های  
\verb!cls!
 آشنایی ندارید، به هیچ وجه به این فایل، یعنی فایل 
\LRE{\verb!tabriz-thesis.cls!}
دست نزنید.

نکته دیگری که باید به آن توجه کنید این است که در فایل 
\LRE{\verb!tabriz-thesis.cls!}،
سه گزینه به نام‌های
\verb!bsc!،
\verb!msc!
و
\verb!phd!
برای تایپ پروژه، پایان‌نامه و رساله،
طراحی شده است. بنابراین اگر قصد تایپ پروژه کارشناسی، پایان‌نامه یا رساله را دارید، 
 در فایل 
\LRE{\verb!tabriz-thesis.tex!}
باید به ترتیب از گزینه‌های
\verb!bsc!،
\verb!msc!
و
\verb!phd!
استفاده کنید. با انتخاب هر کدام از این گزینه‌ها، تنظیمات مربوط به آنها به طور خودکار، اعمل می‌شود.    
\section{مطالب \پ را چطور بنویسم؟}
\subsection{نوشتن فصل‌ها}
همان‌طور که در بخش \ref{sec2} گفته شد، برای جلوگیری از شلوغی و سردرگمی کاربر در هنگام حروف‌چینی، قسمت‌های مختلف \پ از جمله فصل‌ها، در فایل‌های جداگانه‌ای قرار داده شده‌اند. 
بنابراین، اگر می‌خواهید مثلاً مطالب فصل ۱ را تایپ کنید، باید فایل‌های 
\LRE{\verb!tabriz-thesis.tex!}
و
\verb!chapter1!
را باز کنید و محتویات داخل فایل 
\verb!chapter1!
را پاک کرده و مطالب خود را تایپ کنید. توجه کنید که همان‌طور که قبلاً هم گفته شد، تنها فایل قابل اجرا، فایل 
\LRE{\verb!tabriz-thesis.tex!}
است. لذا برای دیدن حاصل (خروجی) فایل خود، باید فایل  
\verb!chapter1!
را 
\verb!Save!
کرده و سپس فایل 
\LRE{\verb!tabriz-thesis.tex!}
را اجرا کنید. یک نکته بدیهی که در اینجا وجود دارد، این است که لازم نیست که فصل‌های \پ را به ترتیب تایپ کنید. می‌توانید ابتدا مطالب فصل ۳ را تایپ کنید و سپس مطالب فصل ۱ را تایپ کنید. 

نکته بسیار مهمی که در اینجا باید گفته شود این است که سیستم \lr{\TeX}، محتویات یک فایل تِک را به ترتیب پردازش می‌کند. به عنوان مثال، اگه فایلی، دارای ۴ خط دستور باشد، ابتدا خط ۱، بعد خط ۲، بعد خط ۳ و در آخر، خط ۴ پردازش می‌شود. بنابراین، اگر مثلاً مشغول تایپ مطالب فصل ۳ هستید، بهتر است
که دو دستور 
\verb!\include{chapter1}!
و
\verb!\include{chapter2}!
را در فایل 
\LRE{\verb!tabriz-thesis.tex!}،
غیرفعال%
\RTLfootnote{
برای غیرفعال کردن یک دستور، کافی است پشت آن، یک علامت
\%
 بگذارید.
}
 کنید. زیرا در غیر این صورت، ابتدا مطالب فصل ۱ و ۲ پردازش شده (که به درد ما نمی‌خورد؛ چون ما می‌خواهیم خروجی فصل ۳ را ببینیم) و سپس مطالب فصل ۳ پردازش می‌شود و این کار باعث طولانی شدن زمان اجرا می‌شود. زیرا هر چقدر حجم فایل اجرا شده، بیشتر باشد، زمان بیشتری هم برای اجرای آن، صرف می‌شود.
\subsection{مراجع}
برای وارد کردن مراجع \پ خود، کافی است فایل 
\verb!references.tex!
را باز کرده و مراجع خود را مانند مراجع داخل آن، وارد کنید. اگر کاربر حرفه‌ای تِک هستید، پیشنهاد می‌شود که از \lr{Bib\TeX} برای 
وارد کردن مراجع استفاده کنید. نکته‌ای که باید به آن توجه کنید این است که در نسخه‌های قدیمی زی‌پرشین، 
قسمت مراجع، حاشیه‌های نامناسبی ایجاد می‌کرد. لذا در نسخه‌های جدید، این حاشیه‌ها اصلاح شده و به خاطر همین، چند دستور جدید، جایگزین شده است. بنابراین، اگه هنوز از نسخه‌های قدیمی زی‌پرشین استفاده می‌کنید، ممکن است هنگام پردازش قسمت مراجع، با خطا مواجه شوید. برای اطلاع از این دستورها، می‌توانید به تالار گفتگوی پارسی‌لاتک و یا راهنمای بسته 
\verb!bidi!
مراجعه کنید.
\subsection{واژه‌نامه فارسی به انگلیسی و برعکس}
برای وارد کردن واژه‌نامه فارسی به انگلیسی و برعکس، چنانچه کاربر مبتدی هستید، بهتر است مانند روش بکار رفته در فایل‌های 
\verb!dicfa2en!
و
\verb!dicen2fa!
عمل کنید. اما چنانچه کاربر پیشرفته هستید، بهتر است از بسته
\verb!glossaries!
استفاده کنید. راهنمای این بسته را می‌توانید به راحتی و با یک جستجوی ساده در اینترنت پیدا کنید.
\subsection{نمایه}
برای وارد کردن نمایه، باید از 
\verb!xindy!
استفاده کنید. زیرا 
\verb!MakeIndex!
با حروف «گ»، «چ»، «پ»، «ژ» و «ک» مشکل دارد و ترتیب الفبایی این حروف را رعایت نمی‌کند. همچنین، فاصله بین هر گروه از کلمات در 
\verb!MakeIndex!،
به درستی رعایت نمی‌شود که باعث زشت شدن حروف‌چینی این قسمت می‌شود. راهنمای چگونگی کار با 
\verb!xindy! 
را می‌توانید در تالار گفتگوی پارسی‌لاتک، پیدا کنید.
\section{اگر سوالی داشتم، از کی بپرسم؟}
برای پرسیدن سوال‌های خود در مورد حروف‌چینی با زی‌پرشین،  می‌توانید به
 \href{http://forum.parsilatex.com}{تالار گفتگوی پارسی‌لاتک}%
\LTRfootnote{\url{http://www.forum.parsilatex.com}}
مراجعه کنید. شما هم می‌توانید روزی به سوال‌های دیگران در این تالار، جواب بدهید.
    
در ادامه، برای فهم بیشتر مطالب، چند تعریف، قضیه و مثال آورده شده است!
\begin{definition}
مجموعه همه ارزیابی‌های  (پیوسته)  روی $(X,\tau)$، دامنه توانی احتمالی
\index{دامنه توانی احتمالی}
$ X $
نامیده می‌شود.
\end{definition}
\begin{theorem}[باناخ-آلااغلو]
\index{قضیه باناخ-آلااغلو}
اگر $ V $ یک همسایگی $ 0 $ در فضای برداری 
\index{فضای!برداری}
 توپولوژیکی $ X $ باشد و 
\begin{equation}\label{eq1}
K=\left\lbrace \Lambda \in X^{*}:|\Lambda x|\leqslant 1 ; \ \forall x\in V\right\rbrace,
\end{equation}
آنگاه $ K $،  ضعیف*-فشرده است که در آن، $ X^{*} $ دوگان
\index{فضای!دوگان}
 فضای برداری توپولوژیکی $ X $ است به ‌طوری که عناصر آن،  تابعی‌های 
خطی پیوسته
\index{تابعی خطی پیوسته}
 روی $X$ هستند.
\end{theorem}
تساوی \eqref{eq1} یکی از مهم‌ترین تساوی‌ها در آنالیز تابعی است که در ادامه، به وفور از آن استفاده می‌شود.
\begin{example}
برای هر فضای مرتب، گردایه 
$$U:=\left\lbrace U\in O: U=\uparrow U\right\rbrace $$
از مجموعه‌های بالایی باز، یک توپولوژی تعریف می‌کند که از توپولوژی اصلی، درشت‌تر  است.
\end{example}
حال تساوی 
\begin{equation}\label{eq2}
\sum_{n=1}^{+\infty} 3^{n}x+70x=\int_{1}^{n}8nx+\exp{(2nx)}
\end{equation}
را در نظر بگیرید. با مقایسه تساوی \eqref{eq2} با تساوی \eqref{eq1} می‌توان نتیجه گرفت که ...!
و
\verb!\chapter{منطق‌های شناختی پویا}
در این فصل سه منطق شناختی پویا را با این رویکرد مطرح می‌کنیم که برای هریک اصولی موسوم به اصول موضوعه‌ی {\reduction\LTRfootnote{\lr{reduction axioms}}} معرفی کرده و با اثبات صحت آنها گامی به سوی تمامیت بر می‌داریم. در انتهای فصل نیز تمامیت را در یک قضیه برای هر سه منطق اثبات خواهیم کرد.
\section{منطق‌های شناختی پویا به منظور به‌روزرسانی غیر احتمالاتی}
منطق‌های شناختی پویا جریان اطلاعات ایجاد شده توسط عمل\LTRfootnote{\lr{event}}‌ها را توصیف می‌کنند. ساده‌ترین عمل آموزنده، و نمونه‌ای رهگشا برای بیشتر این نظریه، اعلان عمومیِ\index{اعلان عمومی} گزاره‌یِ درستی چون $ A $ به گروهی از عامل‌هاست، که به‌صورت $ !A $  نمایش می‌دهیم. به‌روزرسانی برای عمل‌های پیچیده‌تر می‌تواند برحسب «مدل‌های عمل» توصیف شود، که الگوهای پیچیده‌تری از دسترسی عامل‌ها به عملِ در حال رخداد را مدل می‌کنند. پس ابتدا به‌روزرسانی منطق شناختی توسط اعلان عمومی را بررسی می‌کنیم سپس آن را به حالت کلی‌تر، برای هر نوع عمل، توسیع می‌دهیم.

\subsection{منطق اعلان عمومی \texorpdfstring{ \lr{(PAL)}}{(PAL)}}\index{منطق!اعلان عمومی}
تأثیر پویای اعلان عمومی\LTRfootnote{\lr{public announcement}} $ A $ این است که مدل (غیر احتمالاتی) جاری  $ M=(S,\sim,V) $ را به مدل به‌روز شده‌ی $ M|A $ تبدیل می‌کند. این مدل به‌روز شده با تحدید جهان‌های $ M $ به جهان‌هایی که $ A $ در آنها درست است تعریف می‌شود.

اعلان عمومی معمولاً حاوی اطلاعاتی مفید است. از این رو ممکن است که ارزش درستی عبارات شناختی در نتیجه‌ی اعلان تغییر کند. برای مثال قبل از اعلان $ A $ عامل $ a $ آن را نمی‌دانست ولی اکنون می‌داند.

\begin{definition}{\textbf{زبان اعلان عمومی.}}\index{زبان!اعلان عمومی}
زبان اعلان عمومی توسط فرم \lr{Backus-Naur} به‌صورت زیر بیان می‌شود:
\begin{equation*}
\varphi,\psi ::=\ \top\mid\bot\mid p \mid\neg\varphi\mid\varphi\wedge\psi\mid K_i\varphi\mid\left[ !\varphi\right] \psi
\end{equation*}
\end{definition}
فرمول $ [!\varphi]\psi $ به‌صورت «$ \psi $ پس از اعلان $ \varphi $ برقرار است» خوانده می‌شود. زبان بدست آمده  در مدل‌های استاندارد برای منطق شناختی نیز قابل تفسیر است. معناشناسی برای این زبان به غیر از اعلان عمومی همانند تعریف \ref{def3} می‌باشد. معناشناسی اعلان عمومی نیز به‌صورت زیر تعریف می‌شود.
\begin{definition}{\textbf{معناشناسی اعلان عمومی.}}\index{معناشناسی!اعلان عمومی}
فرض کنید مدل شناختی  $ M=(S,\sim,V) $ داده شده باشد و $ s\in S $.
\\

\semanticsb{$ M,s\vDash [!A]\varphi $}{اگر $ M,s\vDash A $ آنگاه $ M|A,s\vDash\varphi $}
\\
\\
که در آن $ M|A $ مدل $ (S',\sim ',V') $ است به طوری که، با فرض
$ \llfloor A\rrfloor =\{t\in S\mid M,t\vDash A\} $:
\begin{LTR}
\begin{itemize}
\item
$ S'=\llfloor A\rrfloor, $
\item
$ \sim'_a=\sim_a\cap(S'\times S'), $
\item
$ V'(p)=V(p)\cap S'. $
\end{itemize}
\end{LTR}
\end{definition}
اصول موضوعه‌‌ی {\reduction} در PAL به‌صورت زیر است:
\begin{align}
&[!A]p\leftrightarrow(A\rightarrow p)\label{1}\\
&[!A]\neg\varphi\leftrightarrow(A\rightarrow\neg[!A]\varphi)\label{2}\\
&[!A](\varphi\wedge\psi)\leftrightarrow([!A]\varphi\wedge[!A]\psi)\label{3}\\
&[!A]K_a \varphi\leftrightarrow(A\rightarrow K_a[!A]\varphi)\label{4}
\end{align}

\begin{theorem}\label{reduct1}\textbf{(صحت اصول موضوعه‌ی  {\reduction} برای اعلان عمومی)}
\end{theorem}
\bp
با ارجاع به هر اصل اثباتی برای آن می‌آوریم.
\begin{itemize}
\item[(\ref{1})]
\begin{align*}
M,s\vDash [!A]p &\ \ \Leftrightarrow\ \ M,s\vDash A\Rightarrow M|A,s\vDash p\tag{1}\\
&\ \ \Leftrightarrow\ \ M,s\vDash A\Rightarrow M,s\vDash p\tag{2}\\
&\ \ \Leftrightarrow\ \ M,s\vDash A\rightarrow p
\end{align*}
اگر $ M,s\vDash A $ آنگاه $ s\in S' $ و اگر $ s\in S' $  آنگاه $ V(p)=V'(p) $. در نتیجه از (1) به (2) و برعکس می‌توان رسید.
\item[(\ref{2})]
\begin{align*}
M,s\vDash[!A]\neg\varphi &\ \ \Leftrightarrow\ \ M,s\vDash A\Rightarrow M|A,s\vDash\neg \varphi\\
&\ \ \Leftrightarrow\ \ M,s\vDash A\Rightarrow (M,s\vDash A\ \textrm{و}\  M|A,s\nvDash\varphi)\\
&\ \ \Leftrightarrow\ \ M,s\vDash A\Rightarrow M,s\vDash\neg[!A]\varphi\\
&\ \ \Leftrightarrow\ \ M,s\vDash A\rightarrow\neg[!A]\varphi
\end{align*}
\end{itemize}
\ep
\subsection{منطق شناختی پویا - به‌روزرسانی مدل‌ها \texorpdfstring{ \lr{(DEL)}}{(DEL)}}\index{منطق!شناختی!پویا}
\begin{definition}{\textbf{مدل عمل\LTRfootnote{\lr{event model}}.}}\index{مدل!عمل}
فرض کنید مجموعه‌ی $ \mA $ از عامل‌ها و زبان منطقی $ \mL $ داده شده باشد، مدل عمل ساختار $ A=(E,\sim,pre) $ است بطوری که
\begin{itemize}
\item
$ E $ مجموعه‌ای متناهی و غیر تهی است از عمل‌ها،
\item
$ \sim $ مجموعه‌ای است از روابط هم‌ارزی $ \sim_a $ روی $ E $ برای هر عامل $ a\in\mA $،
\item
$ pre $ تابعی است که به هر عمل $ e\in E $ فرمولی از $ \mL $ را نسبت می‌دهد.
\end{itemize}
تابع پیش‌شرطِ\LTRfootnote{\lr{precondition function}}\index{تابع پیش‌شرط} $ pre $ با نسبت دادن فرمول $ (pre_e) $ به هر عمل در $ E $ معین می‌کند که در کدام جهان‌ها این عمل‌ها ممکن است روی دهند. این مدل‌ها را مدل به‌روزرسانی نیز می‌نامند.
\end{definition}
این مدل‌ها بسیار شبیه مدل‌های شناختی هستند، با این تفاوت که به‌جای دانش‌های مربوط به وضعیت‌های ثابت، دانش درباره‌ی عمل‌ها \index{عمل}مدل شده است.\RTLfootnote{کلمه‌ی «عمل» ترجمه‌ای است از کلمه‌ی \lr{event}، از آنجایی که این کلمه علاوه بر منطق شناختی پویا در نظریه احتمالات نیز استفاده می‌شود، باید دانست که با تفسیرهای متفاوتی در این دو مقوله به کار می‌رود. در نظریه احتمال، \lr{event} آن است که در منطق بدان گوییم گزاره. در حالی که یک \lr{event} در منطق شناختی پویا به همراه گزاره‌ی پیش‌شرط ایجاد می‌شود، ولی در واقع \lr{event}های مدلِ عمل، مدلِ شناختی داده شده را تغییر می‌دهند و خود بخشی از مدل نیستند. از این پیچیده‌تر، گاهی اوقات به تمام مدل عمل، یک \lr{event} اطلاق می‌شود.} روابط تمییز ناپذیری $ \sim $ روی عمل‌ها ابهام درباره‌ی اینکه چه عملی واقعاً رخ داده است را مدل می‌کنند. $ e\sim_a e' $ می‌تواند به این صورت خوانده شود که «اگر فرض شود که عمل $ e $ رخ داده است رخداد عمل $ e' $ با دانشِ $ a $ سازگار است». 

اعلان عمومی $ [!\varphi] $ نیز خود به نوعی یک مدل عمل است که در آن $ E=\{!\} $ و  $ \sim=\{(!,!)\}  $ و $ pre=\{(!,\varphi)\} $.

نتیجه‌ی رخداد یک عمل نمایش داده شده با $ A $ در وضعیت نمایش داده شده با $ M $ برحسب ساختاری ضربی مدل می‌شود.
\section{منطق شناختی پویای احتمالاتی\texorpdfstring{ \lr{(PDEL)}}{(PDEL)}}\index{منطق!شناختی!پویا!احتمالاتی}
برای اینکه بتوانیم به گونه‌ای صریح و شفاف در باب تغییر داده‌های احتمالاتی در قالبی شناختی-پویا استدلال کنیم، می‌بایست منطق شناختی احتمالاتی موجود را به‌وسیله‌ی اصول موضوعه‌ی {\reduction}‌ مناسب توسعه دهیم. در این بخش نشان می‌دهیم که چگونه می‌توان این کار را بر مبنای معناشناسی مدل‌های عمل احتمالاتی، که معرفی خواهد شد، انجام داد.
\begin{definition}\textbf{زبان شناختی پویای احتمالاتی.}\index{زبان!شناختی!پویا!احتمالاتی}
زبان شناختی پویای احتمالاتی به فرم \lr{Backus-Naur} به‌صورت زیر معرفی می‌شود:

\begin{equation*}
\varphi,\psi ::=\ \top\mid\bot\mid p\mid \neg\varphi\mid\varphi\land\psi\mid K_a\varphi\mid [A,e]\varphi\mid\sum_{i=1}^n r_i \mbP_a(\varphi_i)\geq r
\end{equation*}
با همان نمادگذاری منطق شناختی احتمالاتی، علاوه بر آن $ A $ مدل عمل احتمالاتی و $ e $ عملی از آن می‌باشد. فرمول‌هایی که پیش‌شرط‌ها را در مدل احتمالاتی عمل تعریف می‌کنند از همین زبانی که معرفی شد می‌آیند.

در این زبان علاوه بر خلاصه‌نویسی‌های پیش‌گفته خلاصه‌نویسی‌های زیر نیز مطرح است:

$$\langle A,e \rangle\psi :\quad \neg[A,e]\neg\psi$$
و به منظور اینکه پیش‌شرط‌ها را در یک شئ از زبان فرموله کنیم قرار می‌دهیم
\begin{equation}\label{pg0}
pre_{A,e} :\quad\bigvee_{\varphi\in\Phi,pre(\varphi,e)>0}\varphi
\end{equation}
\end{definition}
\begin{remark}\label{preAe}
در مقاله‌ی \citep{Benthem2009}، $ pre_{A,e} $ به‌صورت زیر مطرح شده است:
\begin{equation}\label{pgeq0}
pre_{A,e} :\quad\bigvee_{\varphi\in\Phi,pre(\varphi,e)\geq 0}\varphi
\end{equation}
این تعریف معادل است با $ \bigvee_{\varphi\in\Phi}\varphi $ زیرا $ pre(\varphi,e) $ تابع احتمال است و همواره بزرگتر یا مساوی صفر است.

به دلایلی که مطرح می‌شود تعریف \ref{pg0} طبیعی‌تر به نظر می‌رسد. اولاً $ pre_{A,e} $ به‌عنوان پیش‌شرط $ e  $ مطرح است پس باید شامل پیش‌شرط‌هایی باشد که به $ e $ احتمال مثبت نسبت می‌دهند، ثانیاً اگر برای هر پیش‌شرط $ \varphi $ داشته باشیم $ pre(\varphi,e)=0 $ می‌توان $ pre_{A,e} $ را تعریف کرد $ \bot $ که از دو جنبه‌ی زیر قابل دفاع است:
\begin{itemize}
\item[-]
از منظر جبری وقتی ترتیب به‌وسیله‌ی استلزام روی فرمول‌ها تعریف شده باشد داریم $ \bot=\bigvee \phi $.
\item[-]
از نقطه نظر منطقی از آنجایی که منظور ما از $ pre(\varphi,e) $ احتمال رخداد $ e $ است وقتی $ \varphi $ برقرار است، زمانی که برای هر $ \varphi\in\Phi $ داریم $ pre(\varphi,e)=0 $، $ e $ از جهت احتمالاتی امکان وقوع ندارد، بنابراین اگر ما $ pre_{A,e} $ را قرار دهیم $ \bot $ از برقراری پیش‌شرط‌های $ e $ جلوگیری به عمل آورده‌ایم و از این رو اجازه نمی‌دهیم $ e $ رخ دهد.
\end{itemize}
\end{remark}!
را در فایل 
\LRE{\verb!tabriz-thesis.tex!}،
غیرفعال%
\RTLfootnote{
برای غیرفعال کردن یک دستور، کافی است پشت آن، یک علامت
\%
 بگذارید.
}
 کنید. زیرا در غیر این صورت، ابتدا مطالب فصل ۱ و ۲ پردازش شده (که به درد ما نمی‌خورد؛ چون ما می‌خواهیم خروجی فصل ۳ را ببینیم) و سپس مطالب فصل ۳ پردازش می‌شود و این کار باعث طولانی شدن زمان اجرا می‌شود. زیرا هر چقدر حجم فایل اجرا شده، بیشتر باشد، زمان بیشتری هم برای اجرای آن، صرف می‌شود.
\subsection{مراجع}
برای وارد کردن مراجع \پ خود، کافی است فایل 
\verb!references.tex!
را باز کرده و مراجع خود را مانند مراجع داخل آن، وارد کنید. اگر کاربر حرفه‌ای تِک هستید، پیشنهاد می‌شود که از \lr{Bib\TeX} برای 
وارد کردن مراجع استفاده کنید. نکته‌ای که باید به آن توجه کنید این است که در نسخه‌های قدیمی زی‌پرشین، 
قسمت مراجع، حاشیه‌های نامناسبی ایجاد می‌کرد. لذا در نسخه‌های جدید، این حاشیه‌ها اصلاح شده و به خاطر همین، چند دستور جدید، جایگزین شده است. بنابراین، اگه هنوز از نسخه‌های قدیمی زی‌پرشین استفاده می‌کنید، ممکن است هنگام پردازش قسمت مراجع، با خطا مواجه شوید. برای اطلاع از این دستورها، می‌توانید به تالار گفتگوی پارسی‌لاتک و یا راهنمای بسته 
\verb!bidi!
مراجعه کنید.
\subsection{واژه‌نامه فارسی به انگلیسی و برعکس}
برای وارد کردن واژه‌نامه فارسی به انگلیسی و برعکس، چنانچه کاربر مبتدی هستید، بهتر است مانند روش بکار رفته در فایل‌های 
\verb!dicfa2en!
و
\verb!dicen2fa!
عمل کنید. اما چنانچه کاربر پیشرفته هستید، بهتر است از بسته
\verb!glossaries!
استفاده کنید. راهنمای این بسته را می‌توانید به راحتی و با یک جستجوی ساده در اینترنت پیدا کنید.
\subsection{نمایه}
برای وارد کردن نمایه، باید از 
\verb!xindy!
استفاده کنید. زیرا 
\verb!MakeIndex!
با حروف «گ»، «چ»، «پ»، «ژ» و «ک» مشکل دارد و ترتیب الفبایی این حروف را رعایت نمی‌کند. همچنین، فاصله بین هر گروه از کلمات در 
\verb!MakeIndex!،
به درستی رعایت نمی‌شود که باعث زشت شدن حروف‌چینی این قسمت می‌شود. راهنمای چگونگی کار با 
\verb!xindy! 
را می‌توانید در تالار گفتگوی پارسی‌لاتک، پیدا کنید.
\section{اگر سوالی داشتم، از کی بپرسم؟}
برای پرسیدن سوال‌های خود در مورد حروف‌چینی با زی‌پرشین،  می‌توانید به
 \href{http://forum.parsilatex.com}{تالار گفتگوی پارسی‌لاتک}%
\LTRfootnote{\url{http://www.forum.parsilatex.com}}
مراجعه کنید. شما هم می‌توانید روزی به سوال‌های دیگران در این تالار، جواب بدهید.
    
در ادامه، برای فهم بیشتر مطالب، چند تعریف، قضیه و مثال آورده شده است!
\begin{definition}
مجموعه همه ارزیابی‌های  (پیوسته)  روی $(X,\tau)$، دامنه توانی احتمالی
\index{دامنه توانی احتمالی}
$ X $
نامیده می‌شود.
\end{definition}
\begin{theorem}[باناخ-آلااغلو]
\index{قضیه باناخ-آلااغلو}
اگر $ V $ یک همسایگی $ 0 $ در فضای برداری 
\index{فضای!برداری}
 توپولوژیکی $ X $ باشد و 
\begin{equation}\label{eq1}
K=\left\lbrace \Lambda \in X^{*}:|\Lambda x|\leqslant 1 ; \ \forall x\in V\right\rbrace,
\end{equation}
آنگاه $ K $،  ضعیف*-فشرده است که در آن، $ X^{*} $ دوگان
\index{فضای!دوگان}
 فضای برداری توپولوژیکی $ X $ است به ‌طوری که عناصر آن،  تابعی‌های 
خطی پیوسته
\index{تابعی خطی پیوسته}
 روی $X$ هستند.
\end{theorem}
تساوی \eqref{eq1} یکی از مهم‌ترین تساوی‌ها در آنالیز تابعی است که در ادامه، به وفور از آن استفاده می‌شود.
\begin{example}
برای هر فضای مرتب، گردایه 
$$U:=\left\lbrace U\in O: U=\uparrow U\right\rbrace $$
از مجموعه‌های بالایی باز، یک توپولوژی تعریف می‌کند که از توپولوژی اصلی، درشت‌تر  است.
\end{example}
حال تساوی 
\begin{equation}\label{eq2}
\sum_{n=1}^{+\infty} 3^{n}x+70x=\int_{1}^{n}8nx+\exp{(2nx)}
\end{equation}
را در نظر بگیرید. با مقایسه تساوی \eqref{eq2} با تساوی \eqref{eq1} می‌توان نتیجه گرفت که ...!
و
\verb!\chapter{منطق‌های شناختی پویا}
در این فصل سه منطق شناختی پویا را با این رویکرد مطرح می‌کنیم که برای هریک اصولی موسوم به اصول موضوعه‌ی {\reduction\LTRfootnote{\lr{reduction axioms}}} معرفی کرده و با اثبات صحت آنها گامی به سوی تمامیت بر می‌داریم. در انتهای فصل نیز تمامیت را در یک قضیه برای هر سه منطق اثبات خواهیم کرد.
\section{منطق‌های شناختی پویا به منظور به‌روزرسانی غیر احتمالاتی}
منطق‌های شناختی پویا جریان اطلاعات ایجاد شده توسط عمل\LTRfootnote{\lr{event}}‌ها را توصیف می‌کنند. ساده‌ترین عمل آموزنده، و نمونه‌ای رهگشا برای بیشتر این نظریه، اعلان عمومیِ\index{اعلان عمومی} گزاره‌یِ درستی چون $ A $ به گروهی از عامل‌هاست، که به‌صورت $ !A $  نمایش می‌دهیم. به‌روزرسانی برای عمل‌های پیچیده‌تر می‌تواند برحسب «مدل‌های عمل» توصیف شود، که الگوهای پیچیده‌تری از دسترسی عامل‌ها به عملِ در حال رخداد را مدل می‌کنند. پس ابتدا به‌روزرسانی منطق شناختی توسط اعلان عمومی را بررسی می‌کنیم سپس آن را به حالت کلی‌تر، برای هر نوع عمل، توسیع می‌دهیم.

\subsection{منطق اعلان عمومی \texorpdfstring{ \lr{(PAL)}}{(PAL)}}\index{منطق!اعلان عمومی}
تأثیر پویای اعلان عمومی\LTRfootnote{\lr{public announcement}} $ A $ این است که مدل (غیر احتمالاتی) جاری  $ M=(S,\sim,V) $ را به مدل به‌روز شده‌ی $ M|A $ تبدیل می‌کند. این مدل به‌روز شده با تحدید جهان‌های $ M $ به جهان‌هایی که $ A $ در آنها درست است تعریف می‌شود.

اعلان عمومی معمولاً حاوی اطلاعاتی مفید است. از این رو ممکن است که ارزش درستی عبارات شناختی در نتیجه‌ی اعلان تغییر کند. برای مثال قبل از اعلان $ A $ عامل $ a $ آن را نمی‌دانست ولی اکنون می‌داند.

\begin{definition}{\textbf{زبان اعلان عمومی.}}\index{زبان!اعلان عمومی}
زبان اعلان عمومی توسط فرم \lr{Backus-Naur} به‌صورت زیر بیان می‌شود:
\begin{equation*}
\varphi,\psi ::=\ \top\mid\bot\mid p \mid\neg\varphi\mid\varphi\wedge\psi\mid K_i\varphi\mid\left[ !\varphi\right] \psi
\end{equation*}
\end{definition}
فرمول $ [!\varphi]\psi $ به‌صورت «$ \psi $ پس از اعلان $ \varphi $ برقرار است» خوانده می‌شود. زبان بدست آمده  در مدل‌های استاندارد برای منطق شناختی نیز قابل تفسیر است. معناشناسی برای این زبان به غیر از اعلان عمومی همانند تعریف \ref{def3} می‌باشد. معناشناسی اعلان عمومی نیز به‌صورت زیر تعریف می‌شود.
\begin{definition}{\textbf{معناشناسی اعلان عمومی.}}\index{معناشناسی!اعلان عمومی}
فرض کنید مدل شناختی  $ M=(S,\sim,V) $ داده شده باشد و $ s\in S $.
\\

\semanticsb{$ M,s\vDash [!A]\varphi $}{اگر $ M,s\vDash A $ آنگاه $ M|A,s\vDash\varphi $}
\\
\\
که در آن $ M|A $ مدل $ (S',\sim ',V') $ است به طوری که، با فرض
$ \llfloor A\rrfloor =\{t\in S\mid M,t\vDash A\} $:
\begin{LTR}
\begin{itemize}
\item
$ S'=\llfloor A\rrfloor, $
\item
$ \sim'_a=\sim_a\cap(S'\times S'), $
\item
$ V'(p)=V(p)\cap S'. $
\end{itemize}
\end{LTR}
\end{definition}
اصول موضوعه‌‌ی {\reduction} در PAL به‌صورت زیر است:
\begin{align}
&[!A]p\leftrightarrow(A\rightarrow p)\label{1}\\
&[!A]\neg\varphi\leftrightarrow(A\rightarrow\neg[!A]\varphi)\label{2}\\
&[!A](\varphi\wedge\psi)\leftrightarrow([!A]\varphi\wedge[!A]\psi)\label{3}\\
&[!A]K_a \varphi\leftrightarrow(A\rightarrow K_a[!A]\varphi)\label{4}
\end{align}

\begin{theorem}\label{reduct1}\textbf{(صحت اصول موضوعه‌ی  {\reduction} برای اعلان عمومی)}
\end{theorem}
\bp
با ارجاع به هر اصل اثباتی برای آن می‌آوریم.
\begin{itemize}
\item[(\ref{1})]
\begin{align*}
M,s\vDash [!A]p &\ \ \Leftrightarrow\ \ M,s\vDash A\Rightarrow M|A,s\vDash p\tag{1}\\
&\ \ \Leftrightarrow\ \ M,s\vDash A\Rightarrow M,s\vDash p\tag{2}\\
&\ \ \Leftrightarrow\ \ M,s\vDash A\rightarrow p
\end{align*}
اگر $ M,s\vDash A $ آنگاه $ s\in S' $ و اگر $ s\in S' $  آنگاه $ V(p)=V'(p) $. در نتیجه از (1) به (2) و برعکس می‌توان رسید.
\item[(\ref{2})]
\begin{align*}
M,s\vDash[!A]\neg\varphi &\ \ \Leftrightarrow\ \ M,s\vDash A\Rightarrow M|A,s\vDash\neg \varphi\\
&\ \ \Leftrightarrow\ \ M,s\vDash A\Rightarrow (M,s\vDash A\ \textrm{و}\  M|A,s\nvDash\varphi)\\
&\ \ \Leftrightarrow\ \ M,s\vDash A\Rightarrow M,s\vDash\neg[!A]\varphi\\
&\ \ \Leftrightarrow\ \ M,s\vDash A\rightarrow\neg[!A]\varphi
\end{align*}
\end{itemize}
\ep
\subsection{منطق شناختی پویا - به‌روزرسانی مدل‌ها \texorpdfstring{ \lr{(DEL)}}{(DEL)}}\index{منطق!شناختی!پویا}
\begin{definition}{\textbf{مدل عمل\LTRfootnote{\lr{event model}}.}}\index{مدل!عمل}
فرض کنید مجموعه‌ی $ \mA $ از عامل‌ها و زبان منطقی $ \mL $ داده شده باشد، مدل عمل ساختار $ A=(E,\sim,pre) $ است بطوری که
\begin{itemize}
\item
$ E $ مجموعه‌ای متناهی و غیر تهی است از عمل‌ها،
\item
$ \sim $ مجموعه‌ای است از روابط هم‌ارزی $ \sim_a $ روی $ E $ برای هر عامل $ a\in\mA $،
\item
$ pre $ تابعی است که به هر عمل $ e\in E $ فرمولی از $ \mL $ را نسبت می‌دهد.
\end{itemize}
تابع پیش‌شرطِ\LTRfootnote{\lr{precondition function}}\index{تابع پیش‌شرط} $ pre $ با نسبت دادن فرمول $ (pre_e) $ به هر عمل در $ E $ معین می‌کند که در کدام جهان‌ها این عمل‌ها ممکن است روی دهند. این مدل‌ها را مدل به‌روزرسانی نیز می‌نامند.
\end{definition}
این مدل‌ها بسیار شبیه مدل‌های شناختی هستند، با این تفاوت که به‌جای دانش‌های مربوط به وضعیت‌های ثابت، دانش درباره‌ی عمل‌ها \index{عمل}مدل شده است.\RTLfootnote{کلمه‌ی «عمل» ترجمه‌ای است از کلمه‌ی \lr{event}، از آنجایی که این کلمه علاوه بر منطق شناختی پویا در نظریه احتمالات نیز استفاده می‌شود، باید دانست که با تفسیرهای متفاوتی در این دو مقوله به کار می‌رود. در نظریه احتمال، \lr{event} آن است که در منطق بدان گوییم گزاره. در حالی که یک \lr{event} در منطق شناختی پویا به همراه گزاره‌ی پیش‌شرط ایجاد می‌شود، ولی در واقع \lr{event}های مدلِ عمل، مدلِ شناختی داده شده را تغییر می‌دهند و خود بخشی از مدل نیستند. از این پیچیده‌تر، گاهی اوقات به تمام مدل عمل، یک \lr{event} اطلاق می‌شود.} روابط تمییز ناپذیری $ \sim $ روی عمل‌ها ابهام درباره‌ی اینکه چه عملی واقعاً رخ داده است را مدل می‌کنند. $ e\sim_a e' $ می‌تواند به این صورت خوانده شود که «اگر فرض شود که عمل $ e $ رخ داده است رخداد عمل $ e' $ با دانشِ $ a $ سازگار است». 

اعلان عمومی $ [!\varphi] $ نیز خود به نوعی یک مدل عمل است که در آن $ E=\{!\} $ و  $ \sim=\{(!,!)\}  $ و $ pre=\{(!,\varphi)\} $.

نتیجه‌ی رخداد یک عمل نمایش داده شده با $ A $ در وضعیت نمایش داده شده با $ M $ برحسب ساختاری ضربی مدل می‌شود.
\section{منطق شناختی پویای احتمالاتی\texorpdfstring{ \lr{(PDEL)}}{(PDEL)}}\index{منطق!شناختی!پویا!احتمالاتی}
برای اینکه بتوانیم به گونه‌ای صریح و شفاف در باب تغییر داده‌های احتمالاتی در قالبی شناختی-پویا استدلال کنیم، می‌بایست منطق شناختی احتمالاتی موجود را به‌وسیله‌ی اصول موضوعه‌ی {\reduction}‌ مناسب توسعه دهیم. در این بخش نشان می‌دهیم که چگونه می‌توان این کار را بر مبنای معناشناسی مدل‌های عمل احتمالاتی، که معرفی خواهد شد، انجام داد.
\begin{definition}\textbf{زبان شناختی پویای احتمالاتی.}\index{زبان!شناختی!پویا!احتمالاتی}
زبان شناختی پویای احتمالاتی به فرم \lr{Backus-Naur} به‌صورت زیر معرفی می‌شود:

\begin{equation*}
\varphi,\psi ::=\ \top\mid\bot\mid p\mid \neg\varphi\mid\varphi\land\psi\mid K_a\varphi\mid [A,e]\varphi\mid\sum_{i=1}^n r_i \mbP_a(\varphi_i)\geq r
\end{equation*}
با همان نمادگذاری منطق شناختی احتمالاتی، علاوه بر آن $ A $ مدل عمل احتمالاتی و $ e $ عملی از آن می‌باشد. فرمول‌هایی که پیش‌شرط‌ها را در مدل احتمالاتی عمل تعریف می‌کنند از همین زبانی که معرفی شد می‌آیند.

در این زبان علاوه بر خلاصه‌نویسی‌های پیش‌گفته خلاصه‌نویسی‌های زیر نیز مطرح است:

$$\langle A,e \rangle\psi :\quad \neg[A,e]\neg\psi$$
و به منظور اینکه پیش‌شرط‌ها را در یک شئ از زبان فرموله کنیم قرار می‌دهیم
\begin{equation}\label{pg0}
pre_{A,e} :\quad\bigvee_{\varphi\in\Phi,pre(\varphi,e)>0}\varphi
\end{equation}
\end{definition}
\begin{remark}\label{preAe}
در مقاله‌ی \citep{Benthem2009}، $ pre_{A,e} $ به‌صورت زیر مطرح شده است:
\begin{equation}\label{pgeq0}
pre_{A,e} :\quad\bigvee_{\varphi\in\Phi,pre(\varphi,e)\geq 0}\varphi
\end{equation}
این تعریف معادل است با $ \bigvee_{\varphi\in\Phi}\varphi $ زیرا $ pre(\varphi,e) $ تابع احتمال است و همواره بزرگتر یا مساوی صفر است.

به دلایلی که مطرح می‌شود تعریف \ref{pg0} طبیعی‌تر به نظر می‌رسد. اولاً $ pre_{A,e} $ به‌عنوان پیش‌شرط $ e  $ مطرح است پس باید شامل پیش‌شرط‌هایی باشد که به $ e $ احتمال مثبت نسبت می‌دهند، ثانیاً اگر برای هر پیش‌شرط $ \varphi $ داشته باشیم $ pre(\varphi,e)=0 $ می‌توان $ pre_{A,e} $ را تعریف کرد $ \bot $ که از دو جنبه‌ی زیر قابل دفاع است:
\begin{itemize}
\item[-]
از منظر جبری وقتی ترتیب به‌وسیله‌ی استلزام روی فرمول‌ها تعریف شده باشد داریم $ \bot=\bigvee \phi $.
\item[-]
از نقطه نظر منطقی از آنجایی که منظور ما از $ pre(\varphi,e) $ احتمال رخداد $ e $ است وقتی $ \varphi $ برقرار است، زمانی که برای هر $ \varphi\in\Phi $ داریم $ pre(\varphi,e)=0 $، $ e $ از جهت احتمالاتی امکان وقوع ندارد، بنابراین اگر ما $ pre_{A,e} $ را قرار دهیم $ \bot $ از برقراری پیش‌شرط‌های $ e $ جلوگیری به عمل آورده‌ایم و از این رو اجازه نمی‌دهیم $ e $ رخ دهد.
\end{itemize}
\end{remark}!
را در فایل 
\LRE{\verb!tabriz-thesis.tex!}،
غیرفعال%
\RTLfootnote{
برای غیرفعال کردن یک دستور، کافی است پشت آن، یک علامت
\%
 بگذارید.
}
 کنید. زیرا در غیر این صورت، ابتدا مطالب فصل ۱ و ۲ پردازش شده (که به درد ما نمی‌خورد؛ چون ما می‌خواهیم خروجی فصل ۳ را ببینیم) و سپس مطالب فصل ۳ پردازش می‌شود و این کار باعث طولانی شدن زمان اجرا می‌شود. زیرا هر چقدر حجم فایل اجرا شده، بیشتر باشد، زمان بیشتری هم برای اجرای آن، صرف می‌شود.
\subsection{مراجع}
برای وارد کردن مراجع \پ خود، کافی است فایل 
\verb!references.tex!
را باز کرده و مراجع خود را مانند مراجع داخل آن، وارد کنید. اگر کاربر حرفه‌ای تِک هستید، پیشنهاد می‌شود که از \lr{Bib\TeX} برای 
وارد کردن مراجع استفاده کنید. نکته‌ای که باید به آن توجه کنید این است که در نسخه‌های قدیمی زی‌پرشین، 
قسمت مراجع، حاشیه‌های نامناسبی ایجاد می‌کرد. لذا در نسخه‌های جدید، این حاشیه‌ها اصلاح شده و به خاطر همین، چند دستور جدید، جایگزین شده است. بنابراین، اگه هنوز از نسخه‌های قدیمی زی‌پرشین استفاده می‌کنید، ممکن است هنگام پردازش قسمت مراجع، با خطا مواجه شوید. برای اطلاع از این دستورها، می‌توانید به تالار گفتگوی پارسی‌لاتک و یا راهنمای بسته 
\verb!bidi!
مراجعه کنید.
\subsection{واژه‌نامه فارسی به انگلیسی و برعکس}
برای وارد کردن واژه‌نامه فارسی به انگلیسی و برعکس، چنانچه کاربر مبتدی هستید، بهتر است مانند روش بکار رفته در فایل‌های 
\verb!dicfa2en!
و
\verb!dicen2fa!
عمل کنید. اما چنانچه کاربر پیشرفته هستید، بهتر است از بسته
\verb!glossaries!
استفاده کنید. راهنمای این بسته را می‌توانید به راحتی و با یک جستجوی ساده در اینترنت پیدا کنید.
\subsection{نمایه}
برای وارد کردن نمایه، باید از 
\verb!xindy!
استفاده کنید. زیرا 
\verb!MakeIndex!
با حروف «گ»، «چ»، «پ»، «ژ» و «ک» مشکل دارد و ترتیب الفبایی این حروف را رعایت نمی‌کند. همچنین، فاصله بین هر گروه از کلمات در 
\verb!MakeIndex!،
به درستی رعایت نمی‌شود که باعث زشت شدن حروف‌چینی این قسمت می‌شود. راهنمای چگونگی کار با 
\verb!xindy! 
را می‌توانید در تالار گفتگوی پارسی‌لاتک، پیدا کنید.
\section{اگر سوالی داشتم، از کی بپرسم؟}
برای پرسیدن سوال‌های خود در مورد حروف‌چینی با زی‌پرشین،  می‌توانید به
 \href{http://forum.parsilatex.com}{تالار گفتگوی پارسی‌لاتک}%
\LTRfootnote{\url{http://www.forum.parsilatex.com}}
مراجعه کنید. شما هم می‌توانید روزی به سوال‌های دیگران در این تالار، جواب بدهید.
    
در ادامه، برای فهم بیشتر مطالب، چند تعریف، قضیه و مثال آورده شده است!
\begin{definition}
مجموعه همه ارزیابی‌های  (پیوسته)  روی $(X,\tau)$، دامنه توانی احتمالی
\index{دامنه توانی احتمالی}
$ X $
نامیده می‌شود.
\end{definition}
\begin{theorem}[باناخ-آلااغلو]
\index{قضیه باناخ-آلااغلو}
اگر $ V $ یک همسایگی $ 0 $ در فضای برداری 
\index{فضای!برداری}
 توپولوژیکی $ X $ باشد و 
\begin{equation}\label{eq1}
K=\left\lbrace \Lambda \in X^{*}:|\Lambda x|\leqslant 1 ; \ \forall x\in V\right\rbrace,
\end{equation}
آنگاه $ K $،  ضعیف*-فشرده است که در آن، $ X^{*} $ دوگان
\index{فضای!دوگان}
 فضای برداری توپولوژیکی $ X $ است به ‌طوری که عناصر آن،  تابعی‌های 
خطی پیوسته
\index{تابعی خطی پیوسته}
 روی $X$ هستند.
\end{theorem}
تساوی \eqref{eq1} یکی از مهم‌ترین تساوی‌ها در آنالیز تابعی است که در ادامه، به وفور از آن استفاده می‌شود.
\begin{example}
برای هر فضای مرتب، گردایه 
$$U:=\left\lbrace U\in O: U=\uparrow U\right\rbrace $$
از مجموعه‌های بالایی باز، یک توپولوژی تعریف می‌کند که از توپولوژی اصلی، درشت‌تر  است.
\end{example}
حال تساوی 
\begin{equation}\label{eq2}
\sum_{n=1}^{+\infty} 3^{n}x+70x=\int_{1}^{n}8nx+\exp{(2nx)}
\end{equation}
را در نظر بگیرید. با مقایسه تساوی \eqref{eq2} با تساوی \eqref{eq1} می‌توان نتیجه گرفت که ...
\clearpage{\pagestyle{empty}\cleardoublepage}
\chapter{منطق‌های شناختی پویا}
در این فصل سه منطق شناختی پویا را با این رویکرد مطرح می‌کنیم که برای هریک اصولی موسوم به اصول موضوعه‌ی {\reduction\LTRfootnote{\lr{reduction axioms}}} معرفی کرده و با اثبات صحت آنها گامی به سوی تمامیت بر می‌داریم. در انتهای فصل نیز تمامیت را در یک قضیه برای هر سه منطق اثبات خواهیم کرد.
\section{منطق‌های شناختی پویا به منظور به‌روزرسانی غیر احتمالاتی}
منطق‌های شناختی پویا جریان اطلاعات ایجاد شده توسط عمل\LTRfootnote{\lr{event}}‌ها را توصیف می‌کنند. ساده‌ترین عمل آموزنده، و نمونه‌ای رهگشا برای بیشتر این نظریه، اعلان عمومیِ\index{اعلان عمومی} گزاره‌یِ درستی چون $ A $ به گروهی از عامل‌هاست، که به‌صورت $ !A $  نمایش می‌دهیم. به‌روزرسانی برای عمل‌های پیچیده‌تر می‌تواند برحسب «مدل‌های عمل» توصیف شود، که الگوهای پیچیده‌تری از دسترسی عامل‌ها به عملِ در حال رخداد را مدل می‌کنند. پس ابتدا به‌روزرسانی منطق شناختی توسط اعلان عمومی را بررسی می‌کنیم سپس آن را به حالت کلی‌تر، برای هر نوع عمل، توسیع می‌دهیم.

\subsection{منطق اعلان عمومی \texorpdfstring{ \lr{(PAL)}}{(PAL)}}\index{منطق!اعلان عمومی}
تأثیر پویای اعلان عمومی\LTRfootnote{\lr{public announcement}} $ A $ این است که مدل (غیر احتمالاتی) جاری  $ M=(S,\sim,V) $ را به مدل به‌روز شده‌ی $ M|A $ تبدیل می‌کند. این مدل به‌روز شده با تحدید جهان‌های $ M $ به جهان‌هایی که $ A $ در آنها درست است تعریف می‌شود.

اعلان عمومی معمولاً حاوی اطلاعاتی مفید است. از این رو ممکن است که ارزش درستی عبارات شناختی در نتیجه‌ی اعلان تغییر کند. برای مثال قبل از اعلان $ A $ عامل $ a $ آن را نمی‌دانست ولی اکنون می‌داند.

\begin{definition}{\textbf{زبان اعلان عمومی.}}\index{زبان!اعلان عمومی}
زبان اعلان عمومی توسط فرم \lr{Backus-Naur} به‌صورت زیر بیان می‌شود:
\begin{equation*}
\varphi,\psi ::=\ \top\mid\bot\mid p \mid\neg\varphi\mid\varphi\wedge\psi\mid K_i\varphi\mid\left[ !\varphi\right] \psi
\end{equation*}
\end{definition}
فرمول $ [!\varphi]\psi $ به‌صورت «$ \psi $ پس از اعلان $ \varphi $ برقرار است» خوانده می‌شود. زبان بدست آمده  در مدل‌های استاندارد برای منطق شناختی نیز قابل تفسیر است. معناشناسی برای این زبان به غیر از اعلان عمومی همانند تعریف \ref{def3} می‌باشد. معناشناسی اعلان عمومی نیز به‌صورت زیر تعریف می‌شود.
\begin{definition}{\textbf{معناشناسی اعلان عمومی.}}\index{معناشناسی!اعلان عمومی}
فرض کنید مدل شناختی  $ M=(S,\sim,V) $ داده شده باشد و $ s\in S $.
\\

\semanticsb{$ M,s\vDash [!A]\varphi $}{اگر $ M,s\vDash A $ آنگاه $ M|A,s\vDash\varphi $}
\\
\\
که در آن $ M|A $ مدل $ (S',\sim ',V') $ است به طوری که، با فرض
$ \llfloor A\rrfloor =\{t\in S\mid M,t\vDash A\} $:
\begin{LTR}
\begin{itemize}
\item
$ S'=\llfloor A\rrfloor, $
\item
$ \sim'_a=\sim_a\cap(S'\times S'), $
\item
$ V'(p)=V(p)\cap S'. $
\end{itemize}
\end{LTR}
\end{definition}
اصول موضوعه‌‌ی {\reduction} در PAL به‌صورت زیر است:
\begin{align}
&[!A]p\leftrightarrow(A\rightarrow p)\label{1}\\
&[!A]\neg\varphi\leftrightarrow(A\rightarrow\neg[!A]\varphi)\label{2}\\
&[!A](\varphi\wedge\psi)\leftrightarrow([!A]\varphi\wedge[!A]\psi)\label{3}\\
&[!A]K_a \varphi\leftrightarrow(A\rightarrow K_a[!A]\varphi)\label{4}
\end{align}

\begin{theorem}\label{reduct1}\textbf{(صحت اصول موضوعه‌ی  {\reduction} برای اعلان عمومی)}
\end{theorem}
\bp
با ارجاع به هر اصل اثباتی برای آن می‌آوریم.
\begin{itemize}
\item[(\ref{1})]
\begin{align*}
M,s\vDash [!A]p &\ \ \Leftrightarrow\ \ M,s\vDash A\Rightarrow M|A,s\vDash p\tag{1}\\
&\ \ \Leftrightarrow\ \ M,s\vDash A\Rightarrow M,s\vDash p\tag{2}\\
&\ \ \Leftrightarrow\ \ M,s\vDash A\rightarrow p
\end{align*}
اگر $ M,s\vDash A $ آنگاه $ s\in S' $ و اگر $ s\in S' $  آنگاه $ V(p)=V'(p) $. در نتیجه از (1) به (2) و برعکس می‌توان رسید.
\item[(\ref{2})]
\begin{align*}
M,s\vDash[!A]\neg\varphi &\ \ \Leftrightarrow\ \ M,s\vDash A\Rightarrow M|A,s\vDash\neg \varphi\\
&\ \ \Leftrightarrow\ \ M,s\vDash A\Rightarrow (M,s\vDash A\ \textrm{و}\  M|A,s\nvDash\varphi)\\
&\ \ \Leftrightarrow\ \ M,s\vDash A\Rightarrow M,s\vDash\neg[!A]\varphi\\
&\ \ \Leftrightarrow\ \ M,s\vDash A\rightarrow\neg[!A]\varphi
\end{align*}
\end{itemize}
\ep
\subsection{منطق شناختی پویا - به‌روزرسانی مدل‌ها \texorpdfstring{ \lr{(DEL)}}{(DEL)}}\index{منطق!شناختی!پویا}
\begin{definition}{\textbf{مدل عمل\LTRfootnote{\lr{event model}}.}}\index{مدل!عمل}
فرض کنید مجموعه‌ی $ \mA $ از عامل‌ها و زبان منطقی $ \mL $ داده شده باشد، مدل عمل ساختار $ A=(E,\sim,pre) $ است بطوری که
\begin{itemize}
\item
$ E $ مجموعه‌ای متناهی و غیر تهی است از عمل‌ها،
\item
$ \sim $ مجموعه‌ای است از روابط هم‌ارزی $ \sim_a $ روی $ E $ برای هر عامل $ a\in\mA $،
\item
$ pre $ تابعی است که به هر عمل $ e\in E $ فرمولی از $ \mL $ را نسبت می‌دهد.
\end{itemize}
تابع پیش‌شرطِ\LTRfootnote{\lr{precondition function}}\index{تابع پیش‌شرط} $ pre $ با نسبت دادن فرمول $ (pre_e) $ به هر عمل در $ E $ معین می‌کند که در کدام جهان‌ها این عمل‌ها ممکن است روی دهند. این مدل‌ها را مدل به‌روزرسانی نیز می‌نامند.
\end{definition}
این مدل‌ها بسیار شبیه مدل‌های شناختی هستند، با این تفاوت که به‌جای دانش‌های مربوط به وضعیت‌های ثابت، دانش درباره‌ی عمل‌ها \index{عمل}مدل شده است.\RTLfootnote{کلمه‌ی «عمل» ترجمه‌ای است از کلمه‌ی \lr{event}، از آنجایی که این کلمه علاوه بر منطق شناختی پویا در نظریه احتمالات نیز استفاده می‌شود، باید دانست که با تفسیرهای متفاوتی در این دو مقوله به کار می‌رود. در نظریه احتمال، \lr{event} آن است که در منطق بدان گوییم گزاره. در حالی که یک \lr{event} در منطق شناختی پویا به همراه گزاره‌ی پیش‌شرط ایجاد می‌شود، ولی در واقع \lr{event}های مدلِ عمل، مدلِ شناختی داده شده را تغییر می‌دهند و خود بخشی از مدل نیستند. از این پیچیده‌تر، گاهی اوقات به تمام مدل عمل، یک \lr{event} اطلاق می‌شود.} روابط تمییز ناپذیری $ \sim $ روی عمل‌ها ابهام درباره‌ی اینکه چه عملی واقعاً رخ داده است را مدل می‌کنند. $ e\sim_a e' $ می‌تواند به این صورت خوانده شود که «اگر فرض شود که عمل $ e $ رخ داده است رخداد عمل $ e' $ با دانشِ $ a $ سازگار است». 

اعلان عمومی $ [!\varphi] $ نیز خود به نوعی یک مدل عمل است که در آن $ E=\{!\} $ و  $ \sim=\{(!,!)\}  $ و $ pre=\{(!,\varphi)\} $.

نتیجه‌ی رخداد یک عمل نمایش داده شده با $ A $ در وضعیت نمایش داده شده با $ M $ برحسب ساختاری ضربی مدل می‌شود.
\section{منطق شناختی پویای احتمالاتی\texorpdfstring{ \lr{(PDEL)}}{(PDEL)}}\index{منطق!شناختی!پویا!احتمالاتی}
برای اینکه بتوانیم به گونه‌ای صریح و شفاف در باب تغییر داده‌های احتمالاتی در قالبی شناختی-پویا استدلال کنیم، می‌بایست منطق شناختی احتمالاتی موجود را به‌وسیله‌ی اصول موضوعه‌ی {\reduction}‌ مناسب توسعه دهیم. در این بخش نشان می‌دهیم که چگونه می‌توان این کار را بر مبنای معناشناسی مدل‌های عمل احتمالاتی، که معرفی خواهد شد، انجام داد.
\begin{definition}\textbf{زبان شناختی پویای احتمالاتی.}\index{زبان!شناختی!پویا!احتمالاتی}
زبان شناختی پویای احتمالاتی به فرم \lr{Backus-Naur} به‌صورت زیر معرفی می‌شود:

\begin{equation*}
\varphi,\psi ::=\ \top\mid\bot\mid p\mid \neg\varphi\mid\varphi\land\psi\mid K_a\varphi\mid [A,e]\varphi\mid\sum_{i=1}^n r_i \mbP_a(\varphi_i)\geq r
\end{equation*}
با همان نمادگذاری منطق شناختی احتمالاتی، علاوه بر آن $ A $ مدل عمل احتمالاتی و $ e $ عملی از آن می‌باشد. فرمول‌هایی که پیش‌شرط‌ها را در مدل احتمالاتی عمل تعریف می‌کنند از همین زبانی که معرفی شد می‌آیند.

در این زبان علاوه بر خلاصه‌نویسی‌های پیش‌گفته خلاصه‌نویسی‌های زیر نیز مطرح است:

$$\langle A,e \rangle\psi :\quad \neg[A,e]\neg\psi$$
و به منظور اینکه پیش‌شرط‌ها را در یک شئ از زبان فرموله کنیم قرار می‌دهیم
\begin{equation}\label{pg0}
pre_{A,e} :\quad\bigvee_{\varphi\in\Phi,pre(\varphi,e)>0}\varphi
\end{equation}
\end{definition}
\begin{remark}\label{preAe}
در مقاله‌ی \citep{Benthem2009}، $ pre_{A,e} $ به‌صورت زیر مطرح شده است:
\begin{equation}\label{pgeq0}
pre_{A,e} :\quad\bigvee_{\varphi\in\Phi,pre(\varphi,e)\geq 0}\varphi
\end{equation}
این تعریف معادل است با $ \bigvee_{\varphi\in\Phi}\varphi $ زیرا $ pre(\varphi,e) $ تابع احتمال است و همواره بزرگتر یا مساوی صفر است.

به دلایلی که مطرح می‌شود تعریف \ref{pg0} طبیعی‌تر به نظر می‌رسد. اولاً $ pre_{A,e} $ به‌عنوان پیش‌شرط $ e  $ مطرح است پس باید شامل پیش‌شرط‌هایی باشد که به $ e $ احتمال مثبت نسبت می‌دهند، ثانیاً اگر برای هر پیش‌شرط $ \varphi $ داشته باشیم $ pre(\varphi,e)=0 $ می‌توان $ pre_{A,e} $ را تعریف کرد $ \bot $ که از دو جنبه‌ی زیر قابل دفاع است:
\begin{itemize}
\item[-]
از منظر جبری وقتی ترتیب به‌وسیله‌ی استلزام روی فرمول‌ها تعریف شده باشد داریم $ \bot=\bigvee \phi $.
\item[-]
از نقطه نظر منطقی از آنجایی که منظور ما از $ pre(\varphi,e) $ احتمال رخداد $ e $ است وقتی $ \varphi $ برقرار است، زمانی که برای هر $ \varphi\in\Phi $ داریم $ pre(\varphi,e)=0 $، $ e $ از جهت احتمالاتی امکان وقوع ندارد، بنابراین اگر ما $ pre_{A,e} $ را قرار دهیم $ \bot $ از برقراری پیش‌شرط‌های $ e $ جلوگیری به عمل آورده‌ایم و از این رو اجازه نمی‌دهیم $ e $ رخ دهد.
\end{itemize}
\end{remark}
\clearpage{\pagestyle{empty}\cleardoublepage}
\chapter{اندازه‌ها و ارزیابی‌ها}
\thispagestyle{empty}
\section{اندازه‌ها و تابعی‌های خطی مثبت روی $\mathrm{C(X)}$}
فرض کنید $X$ یک فضای توپولوژیکی روی ...
\section{تابعی‌های خطی}
در این بخش ...
\clearpage{\pagestyle{empty}\cleardoublepage}
\backmatter
{\small
{\baselineskip=.75cm
\addcontentsline{toc}{chapter}{واژه‌نامه فارسی به انگلیسی}
\thispagestyle{empty}
\chapter*{واژه‌نامه فارسی به انگلیسی}
\markboth{واژه‌نامه فارسی به انگلیسی}{واژه‌نامه فارسی به انگلیسی}

\noindent
\englishgloss{probability}{احتمال}
\englishgloss{posterior probability}{احتمال پسینی}
\englishgloss{prior probability}{احتمال پیشینی}
\englishgloss{occurrence probability}{احتمال رخداد}
\englishgloss{propositional probability}{احتمال گزاره‌ای}
\englishgloss{observation probability}{احتمال {\observation}}
\englishgloss{axiom}{اصل موضوع}
\englishgloss{reduction axioms}{اصول موضوعه‌ی {\reduction}}
\englishgloss{public announcement}{اعلان عمومی}
\englishgloss{probability measure}{اندازه‌ی احتمالاتی}
\englishgloss{inner probability measure}{اندازه‌ی احتمالاتی درونی}
\englishgloss{static}{ایستا}
\englishgloss{belief}{باور}
\englishgloss{update}{به‌روزرسانی}
\clearpage{\pagestyle{empty}\cleardoublepage}
\addcontentsline{toc}{chapter}{واژه‌نامه  انگلیسی به  فارسی}
\thispagestyle{empty}
\chapter*{واژه‌نامه  انگلیسی به  فارسی}
\markboth{واژه‌نامه  انگلیسی به  فارسی}{واژه‌نامه  انگلیسی به  فارسی}

\noindent
\persiangloss{عامل}{agent}
\persiangloss{اصل موضوع}{axiom}
\persiangloss{باور}{belief}
\persiangloss{همه‌دانی مشترک}{common knowledge}
\persiangloss{تمامیت}{completeness}
\persiangloss{سازگار}{consistent}
\persiangloss{جمع‌پذیر شمارا}{countably additive}
\persiangloss{تمییز دادن}{distinguish}
\persiangloss{پویا}{dynamic}
\persiangloss{شناخت}{epistemic}
\persiangloss{عمل}{event}
\persiangloss{یای انحصاری}{exclusive or}
\persiangloss{جمع‌پذیر متناهی}{finitely additive}
\persiangloss{فرمول}{formula}
\persiangloss{فرادانش}{higher order information}
\persiangloss{خودبیمار انگار}{hypochondriac}
\persiangloss{اندازه‌ی احتمالاتی درونی}{inner probability measure}
\persiangloss{خودآگاهی}{intropection}}
\clearpage{\pagestyle{empty}\cleardoublepage}
{\baselineskip=.6cm
\phantomsection
\addcontentsline{toc}{chapter}{نمایه}
\printindex}
\clearpage{\pagestyle{empty}\cleardoublepage}
\phantomsection
\addcontentsline{toc}{chapter}{مراجع}
\bibliographystyle{chicago-fa}
\bibliography{MyReferences}
\clearpage{\pagestyle{empty}\cleardoublepage}
}
%ایجاد صفحه‌ای سفید%%%%%%%%%%%%%%%%%%%%%%%%%%%%%%%%%%%
\newpage
\thispagestyle{empty}
\clearpage
~~~
%%%%%%%%%%%%%%%%%%%%%%%%%%%%%%%%%%%%
% در این فایل، عنوان پایان‌نامه، مشخصات خود و چکیده پایان‌نامه را به انگلیسی، وارد کنید.
% توجه داشته باشید که جدول حاوی مشخصات پایان‌نامه/رساله، به طور خودکار، رسم می‌شود.
%%%%%%%%%%%%%%%%%%%%%%%%%%%%%%%%%%%%
\baselineskip=.6cm
\begin{latin}
\latinuniversity{Shahid Beheshti University}
\latinfaculty{Faculty of Mathematical Sciences}
\latindegree{M. Sc. }
%group:
\latinsubject{Department of Mathematics}
\latinfield{Mathematical Logic}
\latintitle{Probabilistic Dynamic Epistemic Logic}
\firstlatinsupervisor{Dr. Morteza Moniri}
%\secondlatinsupervisor{Second Supervisor}
%\firstlatinadvisor{First Advisor}
%\secondlatinadvisor{Second Advisor}
\latinname{Amir Hossein}
\latinsurname{Sharafi}
\latinthesisdate{2011}
\latinkeywords{epistemic logic, dynamic logic, probabilistic logic, updat, Monty Hall}
\en-abstract{\noindent
In this thesis, we first introduce an instance of probabilistic epistemic logics (PEL) and prove its completeness. Then in our approach toward generalization we will consider a simple model of this logic to develop it to dynamic logics which are able to model information changes in multi-agent systems.
\\
After a short description of non-probabilistic dynamic epistemic logics, we also introduce probabilistic dynamic epistemic logic (PDEL) by taking into account three sources of probability, namely, prior probability of states, occurrence probability for events based on a process corresponding to agents’ view, and the probability of uncertainty of observing events. This three sources are used to provide a generalized update mechanism that is a natural and convenient format for modeling information flow.
\\
Then in order to prove a completeness of probabilistic dynamic epistemic logic from the completeness of probabilistic epistemic logic we present axioms which are sound to reduce the formulas containing dynamic operators to the formulas in the corresponding static language.
\\
Finally we will introduce a kind of probabilistic dynamic epistemic logic which is adapted to solve the Monty Hall dilemma and will prove its completeness. Then we will obtain a formal solution for this dilemma within this logic.
}
\latinvtitle
\end{latin}
\label{LastPage}
\end{document}