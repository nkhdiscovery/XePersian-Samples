\chapter{منطق‌های شناختی ایستا}

\section{منطق شناختی \texorpdfstring{ \lr{(EL)}}{(EL)}}\index{منطق!شناختی}
منطق شناختی\LTRfootnote{\lr{ Epistemic Logic}} 
ابزاری است برای توصيف و تحليل دانش \lr{(knowledge)} يا باور \lr{(belief)} عامل‌ها. اين منطق در ابتدا برای تحليل دانش و كمی بعد برای توصيف باورها به كار گرفته شد\RTLfootnote{
برخی منطق‌ها كه بر باورها تأ كيد دارند منطق \lr{doxastic} ناميده می‌شوند‌، به سبب آنكه ريشه يونانی كلمه‌ی \lr{doxastic} بر شكل‌گيری عقيده اطلاق می‌كند، در حالی كه ريشه يونانی كلمه‌ی \lr{epistemic} بيشتر در زمينه‌ی فهميدن و دانش به كار می‌رود.}. رابطه‌ى بین دانش
\index{دانش}
 و باور
\index{باور}
کمی ظریف است واغلب دانش را باور درست در نظر می‌گیرند.

منطق شناختی با استفاده از زبانی صوری به تحلیل دانش و باور می‌پردازد. ابتدا به‌گونه‌ای غیر صوری منظورمان از «منطق» هر آن چیزی است که تحلیل‌های ما را در بر می‌گیرد، و منظورمان از «زبان»
\index{زبان}
مجموعه‌ی‌ دنباله‌هایی از نمادهاست که توسط قواعد زبانی خاصی ایجاد شده‌اند. اما در ادامه به‌صورتی دقیق‌تر به این مفاهیم می‌پردازیم.

منطقی که اکنون ارائه می‌شود مجموعه‌ی غیر تهی $\mathcal{A}$ از عوامل
\index{عامل}
را دربر دارد. پایه‌ای‌ترین خاصیت‌های جهان درباره‌ی آنچه که عامل\LTRfootnote{\lr{agent}}ها ممکن است استدلال کنند، مجموعه‌ی غیر تهی $\mbbP$ را تشکیل می‌دهد. اعضای $\mbbP$ را گزاره‌های اتمی می‌نامیم.

\section{زبان شناختی پایه‌ای}
\index{زبان!شناختی}
زبان شناختی $ \mL_{EL} $ بر پایه‌ی مجموعه‌ی نمادهای زیر شکل می‌گیرد:

\begin{equation*}
\mbbP\cup\left\{\top,\bot,\neg,\land,(,)\right \} \cup\left\{\square_a : a\in \mathcal{A}\right\}.
\end{equation*}

برای ایجاد رشته‌ای از نمادها که تشکیل فرمول می‌دهند از قواعد نحوی زیر استفاده می‌کنیم.

مجموعه‌ی فرمول‌ها
\index{فرمول}
کوچکترین مجموعه‌ای است که شرایط زیر را داشته باشد:
\begin{itemize}
\item
 نماد $\bot$ و $ \top $
 و همه‌ی عناصر $p\in \mbbP$ فرمول هستند.
\item
اگر $\varphi$ فرمول باشد آنگاه $(\neg\varphi)$ فرمول است، و برای هر $a\in\mathcal{A}$، $(\square_a\varphi)$   فرمول است.
\item
اگر $\varphi$ و $\psi$ فرمول باشند، آنگاه $(\varphi\land\psi)$ نیز فرمول است
\end{itemize}
معمولاً برای راحتی خارجی‌ترین پرانتز را نمی‌نویسیم. نحو اغلب به شیوه‌ای دیگر بیان می‌شود که فرم \lr{Bachus-Naur} نامیده می‌شود. با استفاده از این فرم برای بدست آوردن همان فرمول‌هایی که در بالا به‌صورت استقرایی بیان شد می‌توان اینگونه نوشت:
\begin{equation*}
\varphi,\psi ::=\ \top\mid\bot\mid p\mid \neg\varphi\mid\varphi\land\psi\mid\square_a\varphi
\end{equation*}
نماد $\neg$ و $\land$ معمولاً عملگرهای بولی اطلاق می‌شوند و $\square_a$ برای هر $a\in\mathcal{A}$ عملگرهای شناختی نامیده می‌شوند.

حال معنای شهودی هر فرمول را بیان می‌کنیم. فرمول $ \top $ همیشه درست و $\bot$ همیشه غلط است. هر گزاره‌ی اتمی $p$ خاصیتی پایه‌ای از جهان را بیان می‌کند، که یا درست است و یا غلط. فرمول $\neg\varphi$، «نقیض $\varphi$» و فرمول $(\varphi\land\psi)$، «$\varphi$ و $\psi$» خوانده می‌شود. درنهایت به‌ازای هر عامل $a$، فرمول $\square_a\varphi$ به‌صورت «$a$ باور دارد $\varphi$ را» و یا « $a $ می‌داند $ \varphi $ را» خوانده می‌شود. در هر جهان برای هر فرمول یک ارزش درستی مفروض است.

برای آنکه معنای فرمول‌ها را دقیقتر کنیم، نوعی ساختار را که مدل کریپکی\LTRfootnote{\lr{Kripke Model}} \index{مدل!کریپکی}نامیده می‌شود تعریف می‌کنیم. مدل‌های کریپکی که به اقتباس از \lr{Saul Kripke} نام گرفته‌اند، ساختارهای مرتبطی هستند که در معنا‌شناسی‌های منطق وجهی\index{منطق!وجهی} استفاده می‌شوند. منطق شناخت نوع خاصی از منطق وجهی است. 
\begin{definition}\textbf{مدل‌های کریپکی}
مدل کریپکی $M$ سه‌تایی $(S,\xra{\scr{\mA}},V)$ است، که مؤلفه‌هایش به این صورت توصیف می‌شوند:

$S$ 
مجموعه‌ای است که اعضایش وضعیت‌ها یا جهان‌های ممکن
\index{جهان ممکن}
نامیده می‌شوند. هر جهان ممکن با وضعیت ممکنی که عاملی با آن مواجه است متناظر می‌شود. 

مؤلفه‌ی $\xra{\scr{\mA}}$ مجموعه‌ای است از روابط دوتایی $\xra{a}$ که به‌ازای هر $ a\in\mA $ روی $ S $ تعریف شده‌اند. از نماد $ s\xra{a}t $ استفاده می‌کنیم برای اینکه نشان دهیم $ (s,t)\in\xra{a} $. اگر این روابط دوتایی هم‌ارزی باشند آنها را به‌صورت $ \sim_a $ نمایش می‌دهیم و مجموعه‌ی شامل آنها را $ \sim $ می‌نامیم. معمولاً $ s\sim_a t $ را به این صورت معنا می‌دهند که عامل $a$ جهان $ s $ و $ t $ را از هم تمییز
\index{تمییز دادن}
نمی‌دهد، یعنی هر دو را یک جهان می‌پندارد، و این تحلیلگر است که می‌داند این دو جهان متفاوتند.

در نهایت $V$ تابعی است که هر گزاره‌ی اتمی را به زیرمجموعه‌ای از $ S $ می‌نگارد .هر گزاره‌ی اتمی می‌تواند با گزاره‌ای از زبان طبیعی متناظر شود. به‌عنوان مثال گزاره‌ی $ p $ را می‌توان به‌صورت «بیرون، هوا بارانی است» خواند، و اگر جهان $ s \in S $ در مجموعه‌ی $ V(p) $ باشد، ما $ s $ را یکی از چند جهان ممکنی در نظر می‌گیریم که در آن وضعیت بارانی بودن بیرون برقرار باشد.

کلاس همه‌ی مدل‌های کریپکی را $ \mbbK $ می‌نامیم و مدل‌های کریپکی با رابطه‌ی هم‌ارزی را مدل‌های شناختی 
\index{مدل!شناختی}
می‌نامیم و کلاس این مدل‌ها را به‌صورت $ \mbbS\5 $ نمایش می‌دهیم.
\end{definition}
\section{همه‌دانی مشترک}
در زبانی که ارایه شد، می‌توان با استفاده از فرمول زیر بیان کرد که همه $ \varphi $ را می‌دانند:
\begin{equation*}
 \square\varphi\equiv\bigwedge\left\lbrace \square_a\varphi : a\in\mA \right\rbrace 
\end{equation*}
البته اغلب، این دانش به این وابسته است که آیا عامل‌ها از این دانشِ متقابل آگاهند یا نه. ما می‌توانیم این موضوع را به این صورت بنویسیم که نه‌تنها همه $ \varphi $ را می‌دانند، بلکه همه می‌دانند که همه $ \varphi $ را می‌دانند:
\begin{equation*}
\square\varphi\wedge\square\square\varphi
\end{equation*}
هرچندکه به بیش از درجه‌ی سه استدلال نمی‌شود و ممکن است گفته شود که اگر
\begin{equation*}
\varphi\wedge\square\varphi\wedge\square\square\varphi\wedge\square\square\square\varphi ,
\end{equation*}
آنگاه $ \varphi $ همه‌دانی مشترک\LTRfootnote{common knowledge} است اما ما عامل‌هایی را در نظر می‌گیریم که توانایی استدلال ایده‌آلی دارند و درجات دلخواه را در نظر می‌گیرند. از این رو می‌گوییم که $ \varphi $ همه‌دانی مشترک است اگر و فقط اگر $ \square^n\varphi $ برای هر $ n\geqslant 0 $، جایی که $ \square^0\varphi\equiv\varphi $ و برای هر $ n\geqslant 0 $، $ \square^{n+1}\varphi\equiv\square\square^n\varphi $. می‌توانیم فرمول‌ همه‌دانی مشترک را به فرم $ \square^*\varphi $ به زبان اضافه کنیم، و معناشناسی‌اش را به‌صورت زیر تعریف کنیم:
\begin{itemize}
\item
$ M,s\vDash\square^*\varphi $ اگر و فقط اگر $ M,t\vDash\varphi $ برای هر $ t $ که برای آن $ s\rightarrow^* t $، که در آن $ \rightarrow^* $ به‌صورت بستار تراگذری انعکاسی از رابطه‌ی $ \bigcup\left\lbrace \xra{a}: a\in\mA\right\rbrace $ تعریف شده است.
\end{itemize}

همچنین می‌توانیم به همین منوال برای هر مجموعه از عامل‌های $ \mB\subseteq\mA $ فرمول‌هایی را به شکل $ \square_{\mB}^*\varphi $ اضافه کنیم، که به‌صورت «$ \varphi $ در میان عامل‌های درون $ \mB $ همه‌دانی مشترک است» خوانده می‌شود.  برای تعریف معناشناسی، کافی است در تعریف معناشناسی $\square^*\varphi $، $ \mB $ را جایگزین $ \mA $ کنیم. همچنین توجه کنید که $ \square^*\varphi $ می‌تواند با فرمول $ \square_{\mA}^*\varphi $ جایگزین شود.

\section{اصول موضوعه وقواعد منطق شناختی}
منطق متناظر با اصول موضوعه و قواعد زیر را منطق $ \mK $ می‌نامیم
\\

\leftright{تمامی نمونه جانشین‌های 
همانگو\footnotemark ‌های گزاره‌ای}
{ }{$(\square 1)$}

\LTRfootnotetext{tautology}
\leftright{توزیع $ \square_a $  روی $ \rightarrow $}
{$ \square_a(\varphi\rightarrow\psi)\rightarrow (\square_a\varphi\rightarrow \square_a\psi)  $}{$(\square 2)$}

\leftright{قاعده‌ی وضع مقدم\footnotemark}
{$ \displaystyle\frac{\varphi\quad\quad\varphi\rightarrow\psi}{\psi} $}{\lr{(R1)}\index{قاعده!وضع مقدم}}

\LTRfootnotetext{\lr{modus ponens rule}}
\leftright{قاعده‌ی ضرورت\footnotemark $ \square_a $}
{$ \displaystyle\frac{\varphi}{\square_a\varphi} $}{\lr{(R2)}}\index{قاعده!ضرورت}
\LTRfootnotetext{\lr{necessitation rule}}
\\ \\
\begin{theorem}\textbf{(صحت اصول موضوعه و قواعد منطق شناختی)}
اگر $ \mK\vdash\varphi $ آنگاه $ \mbbK\vDash\varphi $؛ اگر $ \mS\5\vdash\varphi $ آنگاه $ \mbbS\5\vDash\varphi $.
\end{theorem}
\bp
برای اثبات می‌توانید به کتاب \citep{ELMeyer} رجوع کنید.
\ep
\begin{theorem}\textbf{(تمامیت منطق شناختی)}
اگر $ \mbbK\vDash\varphi $ آنگاه $ \mK\vdash\varphi $؛ اگر $ \mbbS\5\vDash\varphi $ آنگاه $ \mS\5\vdash\varphi $.
\end{theorem}
\bp
برای اثبات می‌توانید به کتاب \citep{ELMeyer} رجوع کنید.
\ep
\section{منطق شناختی احتمالاتی \texorpdfstring{ \lr{(PEL)}}{(PEL)}}\index{منطق!شناختی!احتمالاتی}

\begin{definition}{\textbf{زبان‌ شناختی احتمالاتی $\mL_{PEL}$.}}\index{زبان!شناختی!احتمالاتی}
این زبان بر پایه‌ی مجموعه‌ی شمارای $ \mbbP $ از گزاره‌های اتمی، مجموعه‌ی متناهی $ \mA $ از عامل‌ها، عملگر شناختی $ K_a $ و نماد تابعی احتمالاتی $ \mbP_a $ شکل می‌گیرد. فرمول‌های خوش‌تعریف با استفاده از فرم \lr{Backus-Naur} به‌صورت زیر بیان می‌شوند:
\begin{equation*}
\varphi,\psi ::=\ \top\mid\bot\mid p\mid \neg\varphi\mid\varphi\land\psi\mid K_a\varphi\mid \sum_{i=1}^n r_i \mbP_a(\varphi_i)\geq r
\end{equation*}
\end{definition}